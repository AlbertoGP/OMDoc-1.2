%%%%%%%%%%%%%%%%%%%%%%%%%%%%%%%%%%%%%%%%%%%%%%%%%%%%%%%%%%%%%%%%%%%%%%%%%
% This file is part of the LaTeX sources of the OMDoc 1.2 specifiation
% Copyright (c) 2006 Alberto Gonz\'alez Palomo
% This work is licensed by the Creative Commons Share-Alike license
% see http://creativecommons.org/licenses/by-sa/2.5/ for details
\svnInfo $Id: main.tex 6251 2006-12-20 18:45:16Z  $
\svnKeyword $HeadURL: https://svn.omdoc.org/repos/omdoc/branches/omdoc-1.2/doc/spec/projects/qmath/main.tex $
%%%%%%%%%%%%%%%%%%%%%%%%%%%%%%%%%%%%%%%%%%%%%%%%%%%%%%%%%%%%%%%%%%%%%%%%%

\section[QMath Parser]{QMath: A Human-Oriented Language and Batch Formatter for
  OMDoc}

\begin{project}{qmath}{http://www.matracas.org/qmath/index.en.html}
\pauthors{Alberto Gonz\'alez Palomo}
\pinstitute{Toledo, Spain\footnotemark}
\end{project}
\footnotetext{\tiny{The author is currently employed part-time in the {\activemath} project,
    developed by Saarland University and the DFKI, but this work was done on his own,
    without their supervision or support.}}

{\qmath} is a batch processor that produces an {\omdoc} file from a plain {\unicode}
text document, in a similar way to how {\TeX} produces a DVI file from a plain
text source. Its purpose is to allow fast writing of mathematical
documents, using plain text and a straightforward syntax (like in computer algebra
systems) for mathematical expressions.

The ``Q'' was intended to mean ``quick'', since {\qmath} began in 1998 as an abbreviated
notation for {\mathml}. The first version (0.1) just expanded the formulas found enclosed
by ``{\$}'' signs, which were abbreviated forms of the {\mathml} element names, and added
the extra markup such as {\snippet{<mrow>}} and the like. The second (0.2) did the same
thing, but this time allowing an algebraic notation that was fixed in the source code.
Finally, version 0.3 allowed the redefinition of symbols while parsing, but it was not
capable of expanding formulas embedded in XML documents like the previous ones did until
version 0.3.8.\footnote{\footnotesize{This offers an alternative to the {\scsys{OQMath}}
    wrapper mentioned in {\mysecref{activemath}}.}}  For a more detailed history
see~\cite{QMathHistory:URL}.

{\qmath} is very simple: it just parses a text (UTF-8) file according to a user-definable
table of symbols, and builds an XML document from that. The symbol definitions are grouped
in files called ``contexts''. The idea is that when you declare a context, its file is
loaded and from then on these symbol definitions take precedence over any previous one,
thus setting the context for parsing of subsequent expressions.

The grouping of symbols in the context files is arbitrary. However, the
ones included with {\qmath} follow the
{\openmath} Content Dictionaries hierarchy so that,
for instance, the English language syntax for the symbols in
the ``arith1'' CD is defined in the context ``Mathematics/OpenMath/arith1''.

\setbox0=\hbox{\begin{minipage}{10.6cm}\scriptsize
\lstinputlisting[frame=none,numbers=none,nolol]{projects/qmath/thoughtcrime.omdoc}
\end{minipage}}
\begin{myfig}{qmathex}{A minimal {\qmath} document (top left) and its {\omdoc} result (bottom). Some symbol definitions are displayed in the top right.}\scriptsize
\begin{tabular}{|p{5.2cm}|p{5.4cm}|}\hline
{\scriptsize\begin{lstlisting}[numbers=none,frame=none]
QMATH 0.3.8
:en
Context: "Mathematics/OMDoc"

:"Diary"
:W. Smith
:1984-04-04 18:43:00+00:00

Context: "Mathematics/Arithmetic"

Theory:[<-thoughtcrime]

:"Down with Big Brother"
Freedom is the freedom to say $2+2=4$. 
If that is granted, all else follows.
\end{lstlisting}}
&
{\tiny{From {\url{contexts/en/Mathematics/OpenMath/arith1.qmath}}:}}
{\scriptsize\begin{lstlisting}[frame=none,numbers=none,mathescape]
Symbol: plus OP_PLUS "arith1:plus"
Symbol: + OP_PLUS "arith1:plus"
Symbol: sum APPLICATION "arith1:sum"
Symbol: $\Sigma$ APPLICATION "arith1:sum"
 $\cdots$
\end{lstlisting}}
{\tiny{From {\url{contexts/en/Mathematics/OpenMath/relation1.qmath}}:}}
{\scriptsize\begin{lstlisting}[frame=none,numbers=none,mathescape]
Symbol: = OP_EQ "relation1:eq"
Symbol: neq OP_EQ "relation1:neq"
Symbol: $\neg$= OP_EQ "relation1:neq"
Symbol: $\neq$ OP_EQ "relation1:neq"
\end{lstlisting}}
$\cdots$
\\\hline
\multicolumn{2}{|l|}{\box0}\\\hline
\end{tabular}
\end{myfig}

{\Myfigref{qmathex}} shows a minimal {\qmath} document, and the {\omdoc} document
generated from it. The first line ("{\tt{QMATH 0.3.8}}") in the {\qmath} document is
required for the parser to recognize the file. The lines beginning with ``{\tt{:}}'' are
metadata items, the first of which, {\tt{:en}}, declares the primary language
for the document, in this case English.
Specifying the language is required, as it sets the basic keywords
accordingly, and there is no default (privileged) language in {\qmath}.
For example, the English keyword ``{\tt{Context}}'' is written
``{\tt{Contexto}}'' if the language is Spanish.
(Similarly, the arithmetic context is "{\tt{Matem\'aticas/Aritm\'etica}}").
Then, the ``OMDoc'' context is loaded, defining the XML elements to be
produced by the metadata items and the different kinds of paragraphs:
plain text, theorem, definition, proof, example, etc.

After that setup come the document title, author name (one line for each author), and
date, which form the content of the {\omdoc} metadata element.

The document is composed of paragraphs (which can be nested) separated by empty lines, and
formulas are written enclosed by ``{\$}'' signs.

There is an Emacs mode included in the source distribution, that
provides syntax highlighting and basic navigation based on element identifiers.

It is also possible to use it on an XML document for expanding only the mathematical
expressions. {\qmath} will detect automatically the input format, either {\qmath}
text or XML, and in the later case output everything verbatim except for the
{\qmath} language fragments found inside the XML processing instructions of the form
{\tt{<?QMath ... ?>}} and the mathematical expressions between ``{\$}''.
\setbox0=\hbox{\begin{minipage}{10.6cm}\scriptsize
\lstinputlisting[frame=none,numbers=none,nolol]{projects/qmath/thoughtcrime.qmath}
\end{minipage}}
\begin{myfig}{qmathexembedded}{The same example document, using {\qmath} only for the formulas.}\scriptsize
\begin{tabular}{|p{5.2cm}|p{5.4cm}|}\hline
\multicolumn{2}{|l|}{\box0}\\\hline
\end{tabular}
\end{myfig}

While {\qmath} was a good improvement over manual typing of the {\omdoc}
{\xml}, it does not scale well: in real documents, with more than a couple
of nesting levels, it is difficult to keep track of where the current
paragraph belongs.

One solution is to use it only for the mathematical expressions,
and rely on some XML editor for the document navigation and
organization, such as the {\omdoc} mode for Emacs described in
{\mysecref{omdocmode}} or the {\sc OQmath} mode for {\sc jEdit} in {\mysecref{jeditoqmath}}.
Another is to use the {\scsys{Sentido}} browser/editor in {\mysecref{sentido}},
which reimplements and extends {\qmath}'s functionality.

{\qmath} is Free Software distributed under the GNU General Public License (GPL~\cite{GPL}).

%%% Local Variables: 
%%% mode: latex
%%% TeX-master: "../../omdoc"
%%% End: 

% LocalWords:  QMath qmath Gonz alez Palomo DVI mrow OQMath activemath UTF EQ
% LocalWords:  arith nolol qmathex thoughtcrime eq neq Contexto Matem aticas eq
% LocalWords:  Aritm etica metadata qmathexembedded omdocmode OQmath jEdit GPL
% LocalWords:  jeditoqmath Sentido sentido reimplements qmath OQmath sentido
