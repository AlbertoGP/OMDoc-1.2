%%%%%%%%%%%%%%%%%%%%%%%%%%%%%%%%%%%%%%%%%%%%%%%%%%%%%%%%%%%%%%%%%%%%%%%%%
% This file is part of the LaTeX sources of the OMDoc 1.2 specifiation
% Copyright (c) 2006 Serge Autexier, Christoph Benzm\"uller, Armin Fiedler, and Henri Lesourd
% This work is licensed by the Creative Commons Share-Alike license
% see http://creativecommons.org/licenses/by-sa/2.5/ for details
\svnInfo $Id: main.tex 6161 2006-10-03 13:04:46Z  $
\svnKeyword $HeadURL: https://svn.omdoc.org/repos/omdoc/branches/omdoc-1.2/doc/spec/projects/texmacs/main.tex $
%%%%%%%%%%%%%%%%%%%%%%%%%%%%%%%%%%%%%%%%%%%%%%%%%%%%%%%%%%%%%%%%%%%%%%%%%

\section[Proof Assistants in Scientific Editors]{Integrating Proof Assistants as Plugins
  in a Scientific Editor}
\begin{project}{texmacs-omega}{http://www.ags.uni-sb.de/~omega/projects/verimathdoc}
  \pauthors{Serge Autexier, Christoph Benzm\"uller, Armin Fiedler, and Henri Lesourd}
  \pinstitute{Computer Science, Saarland University, Saarbr\"ucken, Germany}
\end{project}

\twin{proof}{assistant}\twin{scientific}{editor}
\index{texmacs@\texmacs}\index{OMEGA@\OMEGA}

In contrast to computer algebra systems (CASs), mathematical {\twintoo{proof
    assistance}{system}s} have not yet achieved considerable recognition and relevance in
{\twintoo{mathematical}{practice}}.  One significant shortcoming of the current systems is
that they are not fully integrated or accessible from within standard
{\twintoo{mathematical}{text-editors}} and that therefore a duplication of the
representation effort is typically required. For purposes such as {\indextoo{tutoring}},
{\indextoo{communication}}, or {\indextoo{publication}}, the mathematical content is in
practice usually encoded using common mathematical representation languages by employing
standard mathematical editors (e.g., {\LaTeX} and {\emacs}). Proof assistants, in
contrast, require fully formal representations and they are not yet sufficiently linked
with these standard mathematical text editors.  Therefore, we have decided to extend the
mathematical text editor {\texmacs}~\cite{VdH01} in order to provide direct access from it
to the mathematics assistance system {\OMEGA}~\cite{OMEGA02,SBA-05-a}. Generally, we aim
at an approach that is not dependent on the particular proof assistant system to be
integrated~\cite{ABFL-05-a}.

{\texmacs}~\cite{VdH01} is a scientific {\indextoo{WYSIWYG}} text editor that provides
professional typesetting and supports authoring with powerful macro definition facilities
like in {\LaTeX}. The internal document format of {\texmacs} is a {\scsys{Scheme}}
S-expression composed of {\texmacs} specific markup enriched by definable macros.  The
full access to the document format together with the possibility to define arbitrary
{\scsys{Scheme}} functions over the S-expressions makes {\texmacs} an appropriate text
editor for an integration with a mathematical assistance system.

The mathematical proof assistance system {\OMEGA} \cite{OMEGA02,SBA-05-a} provides proof
development at a high level of abstraction using knowledge-based proof planning and the
proofs developed in {\OMEGA} can be verbalized in natural language via the proof
explanation system {\prex}~\cite{Fiedler-01-a}.  As the base calculus of {\OMEGA} we use
the {\CORE} calculus~\cite{Aut03,Aut-05-a}, which supports proof development directly at
the {\emph{\twintoo{assertion}{level}}}~\cite{Hu-96-a}, where proof steps are justified in
terms of applications of definitions, lemmas, theorems, or hypotheses (collectively called
{\emph{assertions}}).\index{core@{\CORE}}\index{calculus}\twin{mathematical}{document}

Now, consider a teacher, student, engineer, or mathematician who is about to write a new
mathematical document in {\texmacs}. A first crucial step in our approach is to link this
new document to one or more mathematical theories provided in a
{\atwintoo{mathematical}{knowledge}{repository}}.  By providing such a link the document
is initialized and {\texmacs} macros for the relevant {\twintoo{mathematical}{symbol}s}
are automatically imported; these macros overload the pure syntactical symbols and link
them to {\twintoo{formal}{semantics}}. In a {\texmacs} display mode, where this additional
semantic information is hidden, the user may then proceed in editing
{\twintoo{mathematical}{text}} as usual.  The definitions, lemmas, theorems and especially
their proofs give rise to extensions of the original theory and the writing of some proof
goes along with an interactive proof construction in {\OMEGA}. The semantic annotations
are used to {\emph{automatically}} build up a corresponding
{\twintoo{formal}{representation}} in {\OMEGA}, thus avoiding a duplicated encoding effort
of the mathematical content.  Altogether this allows for the development of mathematical
documents in professional type-setting quality which in addition can be formally validated
by {\OMEGA}, hence obtaining {\emph{\atwintoo{verified}{mathematical}{document}s}}.

Using {\texmacs}'s macro definition features, we encode theory-specific knowledge such as
types, constants, definitions and lemmas in macros.  This allows us to translate new
textual definitions and lemmas into the formal representation, as well as to translate
(partial) textbook proofs into formal (partial) proof plans.

Rather than developing a new user interface for the mathematical assistance system
{\OMEGA}, we adapt {\OMEGA} to serve as a mathematical service provider for {\texmacs}.
The main difference is that instead of providing a user interface only for the existing
interaction means of {\OMEGA}, we extend {\OMEGA} to support requirements that arise in
the preparation of a semi-formal mathematical document. In the following we present some
requirements that we identified to guide our
developments.\atwin{formal}{mathematical}{document}

The mathematical document should be prepared directly in interaction with {\OMEGA}. This
requires that (1) the {\twintoo{semantic}{content}} of the document is accessible for a
{\twintoo{formal}{analysis}} and (2) the interactions in either direction should be
localized and aware of the surrounding context.

To make the document accessible for formal analysis requires the extraction of the
semantic content and its encoding in some semi-formal representation suitable for further
formal processing.  Since current {\twintoo{natural language}{analysis}} technology cannot
yet provide us with the support required for this purpose, we use semantic annotations in
the {\texmacs} document instead. Since these semantic annotations must be provided by the
author, one requirement is to keep the burden of providing the annotations as low as
possible.

Due to their formal nature the representations of mathematical objects, for instance,
definitions or proofs, in existing mathematical assistance systems are very detailed,
whereas mathematicians omit many obvious or easily inferable details in their documents:
there is a big gap between common mathematical language and formal, machine-oriented
representations.  Thus another requirement to interfacing {\texmacs} to {\OMEGA} is to
limit the details that must be provided by the user in the {\texmacs} document to an
acceptable amount.


In order to allow both the user and the proof assistance system to manipulate the
mathematical content of the document we need a common representation format for this pure
mathematical content implemented both in {\texmacs} and in {\OMEGA}. To this end we define
a language $S$, which includes many standard notions known from other specification
languages, such as terms, formulas, symbol declarations, definitions, lemmas and theorems.
The difference to standard specification languages is that our language~$S$ (i) includes a
language for proofs, (ii) provides means to indicate the logical context of different
parts of a document by fitting the narrative structure of documents rather than imposing a
strictly incremental description of theories as used in specification languages, and (iii)
accommodates various aspects of underspecification, that is, formal details that the
writer purposely omitted.  Given the language $S$, we augment the document format of
{\texmacs} by the language $S$.  Thus, if we denote the document format of {\texmacs} by
$T$, we define a {\atwindef{semantic}{document}{format}} $T+S$ as a document format still
accepted by {\texmacs}.

Ideally this format of the documents and especially the {\twintoo{semantic}{annotation}s}
should not be specific to {\OMEGA} in order to enable the combination of the {\texmacs}
extension with other proof assistance systems as well as the development of independent
proof checking tools. However, an abstract language for proofs that is suitable for our
purposes and that allows for underspecification is not yet completely fixed. So far we
support the assertion-level proof construction rules provided by
{\CORE}~\cite{Aut03,Aut-05-a}. Thus, instead of defining a fixed language $S$, we define a
language $S(P)$ parametrized over a language $P$ for proofs and define the document
format based on $S(C)$, where $C$ denotes the proof language of {\CORE}. This format
supports the {\emph{\twintoo{static}{representation}}} of
{\atwintoo{semantically}{annotated}{documents}}, which can be professionally typeset with
{\texmacs}.

The {\texmacs} document $T+S(C)$ and the pure semantic representation $S(C)$ in the proof
assistant must be synchronized. The basic idea here is to synchronize via a diff/patch
mechanism tailored to the tree structure of the {\texmacs} documents. The differences
between two versions $ts_i$ and $ts_{i+1}$ of the document in $T+S(C)$ are compiled into a
patch description $p$ of the corresponding document $s_i$ in $S(C)$ for $ts_i$, such that
the application of $p$ to $s_i$ results in $s_{i+1}$ which corresponds to $ts_{i+1}$. An
analogous diff/patch technique is used to propagate changes performed by the proof
assistant tool to documents in $S(C)$ towards the {\texmacs} document in $T+S(C)$. In
order to enable the translation of the patch descriptions, a key-based protocol is used to
identify the corresponding parts in $T+S(C)$ and $S(C)$.

Beyond this basic synchronization mechanism, we define a language that allows for the
description of specific interactions between {\texmacs} and the proof assistant.  This
language $M$ is a language for structured menus and actions with an
{\twintoo{evaluation}{semantics}} which allows to flexibly compute the necessary
parameters for the commands and directives employed in interaction with the proof
assistants. The {\texmacs} document format $T+S(C)$ is finally extended to $T+S(C)+M$,
where the menus can be attached to arbitrary parts of a document and the changes of the
documents performed either by the author or by the proof assistants are propagated between
$T+S(C)+M$ and $S(C)+M$ via the diff/patch mechanism. Note that this includes also the
adaptation of the menus, which is a necessary prerequisite to support context-sensitive
menus and actions contained therein.

The goal of the proposed integration is to use {\OMEGA} as a context-sensitive reasoning
and verification service accessible from within the first-class mathematical text editor
{\texmacs}, where the proof assistant adapts to the style an author would like to write
his {\twintoo{mathematical}{publication}}, and to hide any irrelevant system peculiarities
from the user.  The communication between {\texmacs} and {\OMEGA} is realized by an
{\omdoc}-based interface language.

Although so far the proofs are tailored to the rules of the {\CORE} system, the
representation language in principle is parametrized over a specific language for proofs.
We currently replace the {\CORE} specific proof languages by some generic notion of
proofs, in order to obtain a generic format for formalized mathematical documents.
Thereby we started from a language for assertion-level proofs with
underspecification~\cite{mkmlanguage,AF-05-a}, which we developed from previous
experiences with tutorial dialogs about mathematical proofs between a computer and
students~\cite{dialog}.  \twin{assertion-level proof}{underspecification}

%%% Local Variables:
%%% mode: latex
%%% TeX-master: "../../omdoc"
%%% End: 

% LocalWords:  texmacs Autexier Benzm uller Fiedler Lesourd Saarbr ucken CASs
