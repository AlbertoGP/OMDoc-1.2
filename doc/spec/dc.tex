%%%%%%%%%%%%%%%%%%%%%%%%%%%%%%%%%%%%%%%%%%%%%%%%%%%%%%%%%%%%%%%%%%%%%%%%%
% This file is part of the LaTeX sources of the OMDoc 1.2 specifiation
% Copyright (c) 2006 Michael Kohlhase
% This work is licensed by the Creative Commons Share-Alike license
% see http://creativecommons.org/licenses/by-sa/2.5/ for details
\svnInfo $Id: dc.tex 8645 2010-08-09 09:22:42Z clange $
\svnKeyword $HeadURL: https://svn.omdoc.org/repos/omdoc/branches/omdoc-1.2/doc/spec/dc.tex $
%%%%%%%%%%%%%%%%%%%%%%%%%%%%%%%%%%%%%%%%%%%%%%%%%%%%%%%%%%%%%%%%%%%%%%%%%

\begin{tchapter}[id=metadata,short=Metadata]{Metadata (Modules {\DCmodule{spec}} and  {\CCmodule{spec}})}

Metadata\index{metadata} is ``{\twintoo{data}{about data}}'' --- in the case of {\omdoc}
data about documents, such as titles, authorship, language usage, or administrative
aspects like modification dates, distribution rights, and identifiers. To accommodate such
data, {\omdoc} offers the {\element{metadata}} element in many places. The most commonly
used metadata standard is the Dublin Core vocabulary, which is supported in some form by
most formats. {\omdoc} uses this vocabulary for compatibility with other metadata
applications and extends it for {\twintoo{document}{management}} purposes in {\omdoc}.
Most importantly {\omdoc} extends the use of metadata from documents to other (even
mathematical) elements and {\twintoo{document}{fragment}s} to ensure a fine-grained
authorship and {\twintoo{rights}{management}}.

\begin{erratum}[reported-by=Michael Kohlhase,date=2007-09-08]{The content Model for
    {\tt{dc:creator}} and {\tt{cd:contributor}} is simple text}
\begin{myfig}{qtmetadata}{Dublin Core Metadata in {\omdoc}}
  \begin{scriptsize}
\begin{tabular}{|>{\tt}l|>{\tt}l|>{\tt}l|>{\tt}l|}\hline
{\rm Element}& \multicolumn{2}{l|}{Attributes\hspace*{2.25cm}} & Content  \\\hline
             & {\rm Req.}  & {\rm Optional}     &           \\\hline\hline
 dc:creator     &  & xml:id, class, style, role    &  text \\\hline
 dc:contributor &  & xml:id, class, style, role    &  text \\hline
 dc:title       &  & xml:lang    &  \llquote{math vernacular}  \\\hline
 dc:subject     &  & xml:lang    &  \llquote{math vernacular}  \\\hline
 dc:description &  & xml:lang    &  \llquote{math vernacular}  \\\hline
 dc:publisher   &  & xml:id, class, style          &  ANY  \\\hline
 dc:date        &  & action, who &  {\twintoo{ISO}{8601}}  \\\hline
 dc:type        &  &             &  {\rm fixed:} "Dataset" {\rm or\ } "Text" \\\hline
 dc:format      &  &             &  {\rm fixed:} "application/omdoc+xml"  \\\hline
 dc:identifier  &  & scheme      &  ANY  \\\hline
 dc:source      &  &             &  ANY  \\\hline
 dc:language    &  &             &  {\twintoo{ISO}{639}} \\\hline
 dc:relation    &  &             &  ANY  \\\hline
 dc:rights      &  &             &  ANY  \\\hline\hline
 \multicolumn{4}{|l|}{for \llquote{math vernacular} see {\mysecref{mtext}}}\\\hline
\end{tabular}
\end{scriptsize}
\end{myfig}
\end{erratum}

In the following we will describe the variant of Dublin Core metadata elements used in
{\omdoc}\footnote{Note that {\omdocv{1.2}} systematically changes the Dublin Core {\xml}
  tags to synchronize with the tag syntax recommended by the Dublin Core Initiative. The
  tags were capitalized in {\omdoc}1.1}.  Here, the {\element{metadata}} element can
contain any number of instances of any Dublin Core elements described below in any
order. In fact, multiple instances of the same element type (multiple
{\element[ns-elt=dc]{creator}} elements for example) can be interspersed with other
elements without change of meaning.  {\omdoc} extends the Dublin Core framework with a set
of roles (from the MARC relator set~\cite{Marc:relators03}) on the authorship elements and
with a rights management system based on the Creative Commons Initiative.

\begin{tsection}[id=dc-elements]{The Dublin Core Elements (Module {\DCmodule{spec}})}
  
  The descriptions in this section are adapted from~\cite{DCMI:dmt03}, and augmented for
  the application in {\omdoc} where necessary. All these elements live in the
  {\twintoo{Dublin Core}{namespace}} \url{http://purl.org/dc/elements/1.1/}, for which we
  traditionally use the {\twintoo{namespace}{prefix}} {\snippetin{dc:}}.\atwin{Dublin
    Core}{namespace}{URI}

\begin{description}
\item[{\element[ns-elt=dc]{title}}] The title of the element --- note that {\omdoc}
  metadata can be specified at multiple levels, not only at the document level, in
  particular, the Dublin Core {\eldef[ns-elt=dc]{title}} element can be given to assign a
  title to a theorem, e.g. the ``Substitution Value Theorem''.
  
  The {\element[ns-elt=dc]{title}} element can contain
  {\twintoo{mathematical}{vernacular}}, i.e. the same content as the {\element{CMP}}
  defined in {\mysecref{mtext}}. Also like the {\element{CMP}} element, the
  {\element[ns-elt=dc]{title}} element has an
  {\attribute[ns-elt=xml,ns-attr=dc]{lang}{title}} attribute that specifies the language
  of the content. Multiple {\element[ns-elt=dc]{title}} elements inside a
  {\element{metadata}} element are assumed to be translations of each other.
\item[{\element[ns-elt=dc]{creator}}] A primary creator or author of the publication.
  Additional contributors whose contributions are secondary to those listed in
  {\eldef[ns-elt=dc]{creator}} elements should be named in
  {\element[ns-elt=dc]{contributor}} elements.  Documents with multiple co-authors should
  provide multiple {\element[ns-elt=dc]{creator}} elements, each containing one author.
  The order of {\element[ns-elt=dc]{creator}} elements is presumed to define the order in
  which the creators' names should be presented.
  
  As markup for names across cultures is still un-standardized, {\omdoc} recommends that
  the content of a {\element[ns-elt=dc]{creator}} element consists in a single name (as it
  would be presented to the user). The {\element[ns-elt=dc]{creator}} element has an
  optional attribute {\attribute[ns-elt=xml,ns-attr=dc]{id}{creator}} so that it can be
  {\indextoo{cross-reference}d} and a {\attributeshort{role}} attribute to further
  classify the concrete contribution to the element. We will discuss its values in
  {\mysecref{dc-roles}}.
\item[{\element[ns-elt=dc]{contributor}}] A party whose contribution to the publication is
  secondary to those named in {\element[ns-elt=dc]{creator}} elements.  Apart from the
  significance of contribution, the semantics of the {\eldef[ns-elt=dc]{contributor}} is
  identical to that of {\element[ns-elt=dc]{creator}}, it has the same restriction content
  and carries the same attributes plus a
  {\attribute[ns-elt=xml,ns-attr=dc]{lang}{contributor}} attribute that specifies the
  target language in case the contribution is a translation.
\item[{\element[ns-elt=dc]{subject}}] This element contains an arbitrary phrase or keyword,
  the attribute {\attribute[ns-elt=xml,ns-attr=dc]{lang}{subject}} is used for the
  language. Multiple instances of the {\eldef[ns-elt=dc]{subject}} element are supported
  per {\attribute[ns-elt=xml,ns-attr=dc]{lang}{subject}} for multiple keywords.
\item[{\element[ns-elt=dc]{description}}] A text describing the containing element's
  content; the attribute {\attribute[ns-elt=xml,ns-attr=dc]{lang}{description}} is used
  for the language. As description of mathematical objects or {\omdoc} fragments may
  contain formulae, the content of this element is of the form
  ``{\twintoo{mathematical}{text}}'' described in {\mychapref{mtxt}}.  The
  {\eldef[ns-elt=dc]{description}} element is only recommended for {\element{omdoc}}
  elements that do not have a {\element{CMP}} group (see {\mysecref{mtext}}), or if the
  description is significantly shorter than the one in the {\element{CMP}s} (then it can
  be used as an {\indextoo{abstract}}).
\item[{\element[ns-elt=dc]{publisher}}] The entity for making the document available in
  its present form, such as a publishing house, a university department, or a corporate
  entity. The {\eldef[ns-elt=dc]{publisher}} element only applies if the
  {\element{metadata}} is a direct child of the root element ({\element{omdoc}}) of a
  document\twin{document}{root}.
\item[{\element[ns-elt=dc]{date}}] The date and time a certain action was performed on the
  element that contains this. The content is in the format defined by {\xml} Schema data
  type {\snippetin{date\-Time}} (see~\cite{BirMal:XMLSchema:Datatypes} for a discussion),
  which is based on the {\atwintoo{ISO}{8601}{norm}} for dates and times.

  Concretely, the format is
  {\snippet{\llquote{YYYY}-\llquote{MM}-\llquote{DD}T\llquote{hh}:\llquote{mm}:\llquote{ss}}}
  where {\llquote{YYYY}} represents the year, {\llquote{MM}} the month, and {\llquote{DD}}
  the day, preceded by an optional leading ``{\snippet{-}}'' sign to indicate a negative
  number. If the sign is omitted, ``{\snippet{+}}'' is assumed.  The letter
  ``{\snippet{T}}'' is the date/time separator and {\llquote{hh}}, {\llquote{mm}},
  {\llquote{ss}} represent hour, minutes, and seconds respectively.  Additional digits can
  be used to increase the precision of fractional seconds if desired, i.e the format
  {\snippet{\llquote{ss}.\llquote{sss\ldots}}} with any number of digits after the decimal
  point is supported.  The {\eldef[ns-elt=dc]{date}} element has the attributes
  {\attribute[ns-elt=dc]{action}{date}} and {\attribute[ns-elt=dc]{who}{date}} to specify
  who did what: The value of {\attribute[ns-elt=dc]{who}{date}} is a reference to a
  {\element[ns-elt=dc]{creator}} or {\element[ns-elt=dc]{contributor}} element and
  \erratumReplace[reported-by=Christoph Lange,date=2010-08-09]{wrong attribute name}{{\attribute[ns-elt=action]{dc}{date}}}{{\attribute[ns-elt=dc]{action}{date}}} is a keyword for the action
  undertaken. Recommended values include the short forms
  {\attval[ns-elt=dc]{updated}{action}{date}},
  {\attval[ns-elt=dc]{created}{action}{date}},
  {\attval[ns-elt=dc]{imported}{action}{date}},
  {\attval[ns-elt=dc]{frozen}{action}{date}},
  {\attval[ns-elt=dc]{review-on}{action}{date}},
  {\attval[ns-elt=dc]{normed}{action}{date}} with the obvious meanings. Other actions may
  be specified by {\indextoo{URI}s} pointing to documents that explain the action.
\item[{\element[ns-elt=dc]{type}}] Dublin Core defines a vocabulary for the document types
  in {\cite{DCMI:dtv03}}. The best fit values for {\omdoc} are
  \begin{description}
  \item[{\snippetin{Dataset}}]\index{Dataset@{\snippet{Dataset}} as Dublin Core Type}
    defined as ``{\emph{information encoded in a defined structure (for example lists,
      tables, and databases), intended to be useful for direct machine processing}}.''
  \item[{\snippetin{Text}}]\index{Text@{\snippet{Text}} as Dublin Core Type} defined as
    ``{\emph{a resource whose content is primarily words for reading. For example -- books,
      letters, dissertations, poems, newspapers, articles, archives of mailing lists. Note
      that facsimiles or images of texts are still of the genre text.}}''
  \item[{\snippetin{Collection}}]\index{Collection@{\snippet{Collection} as Dublin Core
        Type}} defined as ``{\emph{an aggregation of items. The term collection means that
      the resource is described as a group; its parts may be separately described and
      navigated}}''.
  \end{description}
  The more appropriate should be selected for the element that contains the
  {\eldef[ns-elt=dc]{type}}. If it consists mainly of formal mathematical formulae, then
  {\snippetin{Dataset}} is better, if it is mainly given as text, then {\snippetin{Text}}
  should be used. More specifically, in {\omdoc} the value {\snippetin{Dataset}} signals
  that the order of children in the parent of the {\element{metadata}} is not relevant to
  the meaning. This is the case for instance in formal developments of mathematical
  theories, such as the specifications in {\mychapref{complex-theories}}.
\item[{\element[ns-elt=dc]{format}}] The physical or digital manifestation of the
  resource.  Dublin Core suggests using {\twintoo{MIME}{type}s}~\cite{FreBor:MIME96}.
  Following~\cite{MurLau:xmt01} we fix the content of the {\eldef[ns-elt=dc]{format}}
  element to be the string {\snippet{application/omdoc+xml}} as the {\twintoo{MIME}{type}}
  for {\omdoc}.
\item[{\element[ns-elt=dc]{identifier}}] A string or number used to uniquely identify the
  element.  The {\eldef[ns-elt=dc]{identifier}} element should only be used
  for public identifiers like {\indextoo{ISBN}} or {\indextoo{ISSN}} numbers. The
  numbering scheme can be specified in the {\attribute[ns-elt=dc]{scheme}{identifier}}
  attribute.
\item[{\element[ns-elt=dc]{source}}] Information regarding a prior resource from which the
  publication was derived. We recommend using either a {\indextoo{URI}} or a scientific
  reference including identifiers like ISBN numbers for the content of the
  {\eldef[ns-elt=dc]{source}} element.
\item[{\element[ns-elt=dc]{relation}}] Relation of this document to others.  The content
  model of the {\eldef[ns-elt=dc]{relation}} element is not specified in the {\omdoc}
  format.
\item[{\element[ns-elt=dc]{language}}] If there is a primary language of the document or
  element, this can be specified here. The content of the {\eldef[ns-elt=dc]{language}}
  element must be an {\atwintoo{ISO}{639}{norm}} two-letter language specifier, like
  {\snippetin{en}}$\;\widehat=\;$English, {\snippetin{de}}$\;\widehat=\;$German,
  {\snippetin{fr}}$\;\widehat=\;$French, {\snippetin{nl}}$\;\widehat=\;$Dutch, \ldots.
\item[{\element[ns-elt=dc]{rights}}] Information about rights held in and over the
  document or element content or a reference to such a statement. Typically, a
  {\eldef[ns-elt=dc]{rights}} element will contain a rights management statement, or
  reference a service providing such information. {\element[ns-elt=dc]{rights}}
  information often encompasses Intellectual Property rights (IPR), Copyright, and various
  other property rights. If the {\element[ns-elt=dc]{rights}} element is absent (and no
  {\element[ns-elt=dc]{rights}} information is inherited), no assumptions can be made
  about the status of these and other rights with respect to the document or element.
  
  {\omdoc} supplies specialized elements for the Creative Commons licenses to support the
  sharing of mathematical content. We will discuss them in {\mysecref{creativecommons}}.
\end{description}
Note that Dublin Core also defines a {\oldelement{Coverage}{1.1}} element that
specifies the place or time which the publication's contents addresses. This does
not seem appropriate for the mathematical content of {\omdoc}, which is largely
independent of time and geography.
\end{tsection}

\begin{tsection}[id=dc-roles]{Roles in Dublin Core Elements}

Because the Dublin Core metadata fields for {\element[ns-elt=dc]{creator}} and
{\element[ns-elt=dc]{contributor}} do not distinguish roles of specific parties (such as
author, editor, and illustrator), we will follow the {\indextoo{Open eBook}}
specification~\cite{OpenEBook:oeps99} and use an optional
{\attribute[ns-elt=dc]{role}{*}} attribute for this purpose, which is
adapted for {\omdoc} from the MARC relator code list~\cite{Marc:relators03}.
\begin{description}
\item[{\attval[ns-elt=dc]{aut}{role}{*}}] ({\indextoo{author}}) Use for a
  person or corporate body chiefly responsible for the intellectual
  content of an element. This term may also be used when more than one person or body
  bears such responsibility.
\item[{\attval[ns-elt=dc]{ant}{role}{*}}] (scientific/bibliographic
  antecedent\twin{bibliographic}{antecedent}\twin{scientific}{antecedent}) Use
for the author responsible for a work upon which the element is based.
\item[{\attval[ns-elt=dc]{clb}{role}{*}}] ({\indextoo{collaborator}}) Use
  for a person or corporate body that takes a limited part in the elaboration of a
  work of another author or that brings complements (e.g., appendices, notes) to
  the work of another author.
\item[{\attval[ns-elt=dc]{edt}{role}{*}}] ({\indextoo{editor}}) Use for a
  person who prepares a document not primarily his/her own for publication, such
  as by elucidating text, adding introductory or other critical matter, or
  technically directing an editorial staff.
\item[{\attval[ns-elt=dc]{ths}{role}{*}}] ({\twintoo{thesis}{advisor}}) Use for the person under
  whose supervision a degree candidate develops and presents a thesis, memoir, or text of
  a dissertation.
\item[{\attval[ns-elt=dc]{trc}{role}{*}}] ({\indextoo{transcriber}}) Use
  for a person who prepares a handwritten or typewritten copy from original
  material, including from dictated or orally recorded material. This is also the
  role (on the {\element[ns-elt=dc]{creator}} element) for someone who prepares the {\omdoc}
  version of some mathematical content.
\item[{\attval[ns-elt=dc]{trl}{role}{*}}] ({\indextoo{translator}}) Use
  for a person who renders a text from one language into another, or from an older
  form of a language into the modern form. The target language can be specified by
  {\attribute[ns-elt=xml,ns-attr=dc]{lang}{*}}.
\end{description}
As {\omdoc} documents are often used to formalize existing mathematical texts for use in
mechanized reasoning and computation systems, it is sometimes subtle to specify
authorship.  We will discuss some typical examples to give a guiding intuition.
{\mylstref{sec-edt}} shows metadata for a situation where editor $R$ gives the sources
(e.g. in {\LaTeX}) of an element written by author $A$ to secretary $S$ for conversion
into {\omdoc} format.
\begin{lstlisting}[label=lst:sec-edt,mathescape,
  caption={A Document with Editor ({\snippet{edt}}) and  Transcriber ({\snippet{trc}})},
  index={metadata,dc:title,dc:creator,dc:contributor}]
<metadata>
  <dc:title>The Joy of Jordan $C\sp{*}$ Triples</dc:title>
  <dc:creator role="aut">$A$</dc:creator>
  <dc:contributor role="edt">$R$</dc:contributor>
  <dc:contributor role="trc">$S$</dc:contributor>
</metadata>
\end{lstlisting}

In {\mylstref{formalize}} researcher $R$ formalizes the theory of natural numbers
following the standard textbook $B$ (written by author $A$). In this case we
recommend the first declaration for the whole document and the second one for
specific math elements, e.g. a definition inspired by or adapted from one in book
$B$.

\begin{lstlisting}[label=lst:formalize,mathescape,
  caption={A Formalization with Scientific Antecedent ({\snippet{ant}})},
  index={metadata,dc:title,dc:creator}]
<omdoc xml:id="NNat" version="1.2" xmlns:dc="http://purl.org/dc/elements/1.1/">
  <metadata><dc:title>Natural Numbers</dc:title></metadata>                              
  $\ldots$
  <theory xml:id="NNat.thy">
    <metadata>
      <dc:title>Natural Numbers</dc:title>
      <dc:creator role="aut">$R$</dc:creator>
      <dc:contributor role="ant">$A$</dc:contributor>
      <dc:source>$B$</dc:source>
    </metadata>
    $\ldots$
  </theory>
  $\ldots$
</omdoc>
\end{lstlisting}
\end{tsection}

\begin{tsection}[id=creativecommons,short=Managing Rights]{Managing Rights by Creative
    Commons Licenses (Module {\CCmodule{spec}})}

  The Dublin Core vocabulary provides the {\element[ns-elt=dc]{rights}} element for
  information about rights held in and over the document or element content, but leaves
  the content model unspecified. While it is legally sufficient to describe this
  information in natural language, a content markup format like {\omdoc} should support a
  machine-understandable format. As one of the purposes of the {\omdoc} format is to
  support the sharing and re-use of mathematical content, {\omdoc} provides markup for
  rights management via the {Creative Commons\twin{Creative Commons}{license}}
  (CC{\twin{CC}{license}}) licenses.  Digital rights
  management\atwin{Digital}{rights}{management} (\indextoo{DRM}) and licensing of
  {\twintoo{intellectual}{property}} has become a hotly debated topic in the last
  years. We feel that the {\twintoo{Creative Commons}{license}s} that encourage sharing of
  content and enhance the (scientific) public domain while giving authors some control
  over their intellectual property establish a good middle ground. Specifying rights is
  important, since in the absence of an explicit or implicit (via inheritance)
  {\element[ns-elt=dc]{rights}} element no assumptions can be made about the status of the
  document or fragment.  Therefore {\omdoc} adds another child to the {\element{metadata}}
  element.  This {\eldef[ns-elt=cc]{license}} element is a symbolic representation of the
  Creative Commons legal framework, adapted to the {\omdoc} setting: The Creative Commons
  Metadata Initiative specifies various ways of embedding {\twintoo{CC}{metadata}} into
  documents and {\twintoo{electronic}{artefacts}} like {\indextoo{picture}s} or
  {\twintoo{MP3}{recording}s}. As {\omdoc} is a source format, from which various
  presentation formats are generated, we need a content representation of the CC metadata
  from which the end-user representations for the respective formats can be generated.

\begin{myfig}{cctable}{The {\omdoc} Elements for Creative Commons Metadata}
\begin{scriptsize}
\begin{tabular}{|>{\tt}l|>{\tt}l|>{\tt}p{2.2truecm}|>{\tt}l|}\hline
{\rm Element}& \multicolumn{2}{l|}{Attributes\hspace*{2.25cm}} & Content  \\\hline
             & {\rm Req.}  & {\rm Optional}    &          \\\hline\hline
 cc:license      & & jurisdiction    &  permissions, prohibitions, requirements  \\\hline
 cc:permissions  & & reproduction, distribution, derivative\_works & EMPTY\\\hline
 cc:prohibitions & & commercial\_use & EMPTY \\\hline
 cc:requirements & & notice, copyleft,  attribution & EMPTY \\\hline
\end{tabular}
\end{scriptsize}
\end{myfig}

The Creative Commons Metadata Initiative~\cite{URL:creativecommons} divides the license
characteristics in three types: {\defin{permissions}}, {\defin{prohibitions}} and
{\defin{requirements}}, which are represented by the three elements, which can occur as
children of the {\element[ns-elt=cc]{license}} element. The {\element[ns-elt=cc]{license}}
element has two optional argument:
\begin{description}
\item[{\attribute[ns-elt=cc]{jurisdiction}{license}}] which allows to specify the country
  in whose jurisdiction the license will be enforced\footnote{The {\twintoo{Creative
        Commons}{Initiative}} is currently in the process of adapting their licenses to
    jurisdictions other than the USA, where the licenses
    originated. See~\cite{URL:creativecommonsworldwide} for details and to check for
    progress.}. It's value is one of the {\twintoo{top-level}{domain}} codes of the
  ``Internet Assigned Names Authority (IANA)''~\cite{IANA:TLD}. If this attribute is
  absent, then the original US version of the license is assumed.
\item[{\attribute[ns-elt=cc]{version}{license}}] which allows to specify the version of the
  license. If the attribute is not present, then the newest released version is assumed
  (version 2.0 at the time of writing this {\report})
\end{description}

The following three empty elements can occur as children of the
{\element[ns-elt=cc]{license}} element; their attribute specify the rights bestowed on the
user by the license.  All these elements have the {\twin{Creative Commons}{namespace}}
namespace \url{http://creativecommons.org/ns}\atwin{Creative Commons}{namespace}{URI},
for which we traditionally use the {\twintoo{namespace}{prefix}} {\snippetin{cc:}}.

\begin{itemize}
\item {\eldef[ns-elt=cc]{permissions}} are the rights granted by the license, to model
  them the element has three attributes, which can have the values
  {\attvalveryshort{permitted}} (the permission is granted by the license) and
  {\attvalveryshort{prohibited}} (the permission isn't):
  \begin{center}\scriptsize
    \begin{tabular}{|l|p{6truecm}|>{\tt}l|}\hline
      Attribute & Permission & Default\\\hline\hline
      {\attribute[ns-elt=cc]{reproduction}{permissions}} 
      & the work may be reproduced & permitted\\\hline
      {\attribute[ns-elt=cc]{distribution}{permissions}}  
      & the work may be distributed, publicly displayed, and
      publicly performed & permitted \\\hline
      {\attribute[ns-elt=cc]{derivative\_works}{permissions}}  
      & derivative works may be created and reproduced & permitted \\\hline
    \end{tabular}
  \end{center}
\item {\eldef[ns-elt=cc]{prohibitions}} are the things the license prohibits.
  \begin{center}\scriptsize
    \begin{tabular}{|l|p{6truecm}|>{\tt}l|}\hline
      Attribute & Prohibition & Default\\\hline\hline
      {\attribute[ns-elt=cc]{commercial\_use}{permission}} 
      &  stating that rights may be exercised for commercial purposes.
      & permitted \\\hline
    \end{tabular}
  \end{center}
\item {\eldef[ns-elt=cc]{requirements}} are restrictions imposed by the license.
    \begin{center}\scriptsize
      \begin{tabular}{|l|p{6.5truecm}|>{\tt}l|}\hline
      Attribute & Requirement & Default\\\hline\hline
      {\attribute[ns-elt=cc]{notice}{requirements}}  
      & copyright and license notices must be kept intact & required \\\hline
      {\attribute[ns-elt=cc]{attribution}{requirements}}  
      & credit must be given to copyright holder and/or author & required\\\hline
      {\attribute[ns-elt=cc]{copyleft}{requirements}}  
      & derivative works, if authorized, must be licensed under the same terms as
      the work & required \\\hline
    \end{tabular}
  \end{center}
\end{itemize}

This vocabulary is directly modeled after the Creative Commons
Metadata~\cite{URL:creativecommonsMetadata} which defines the meaning, and provides an
{\rdf}~\cite{LasSwi:rdf99} based implementation. As we have discussed in
{\mysecref{metadata}}, {\omdoc} follows an approach that specifies metadata in the
document itself; thus we have provided the elements described here. In contrast to many
other situations in {\omdoc}, the rights model is not extensible, since only the current
model is backed by legal licenses provided by the creative commons initiative.

{\Mylstref{ccc-copyleft}} specifies a license grant using the Creative Commons
``share-alike'' license: The copyright is retained by the author, who licenses the content
to the world, allowing others to reproduce and distribute it without restrictions as long
as the copyright notice is kept intact. Furthermore, it allows others to create derivative
works based on the content as long as it attributes the original work of the author and
licenses the derived work under the identical license (i.e. the Creative Commons
``share-alike'' as well).
\begin{lstlisting}[label=lst:ccc-copyleft,caption={A Creative Commons License},
  index={metadata,dc:rights,license,permissions,reproduction,distribution,
         derivative_works,prohibitions,commercial_use,requirements,
         notice,copyleft,attribution}]
<metadata>
  <dc:rights>Copyright (c) 2004 Michael Kohlhase</dc:rights>
  <license jurisdiction="de" xmlns="http://creativecommons.org/ns">
    <permissions reproduction="permitted" distribution="permitted" 
                 derivative_works="permitted"/>
    <prohibitions commercial_use="permitted"/>
    <requirements notice="required" copyleft="required" attribution="required"/>
  </license>
</metadata>
\end{lstlisting}
\end{tsection}

\begin{tsection}[id=inheritance]{Inheritance of Metadata}
  
  The {\element{metadata}} elements can be added to many of the {\omdoc} elements,
  including grouping elements that can contain others that contain
  {\element{metadata}}. To avoid duplication, {\omdoc} assumes a
  {\indextoo{priority-union}} semantics for the Dublin Core elements
  {\element[ns-elt=dc]{creator}}, {\element[ns-elt=dc]{contributor}},
  {\element[ns-elt=dc]{date}}, {\element[ns-elt=dc]{type}}, {\element[ns-elt=dc]{format}},
  {\element[ns-elt=dc]{source}}, {\element[ns-elt=dc]{language}}, and
  {\element[ns-elt=dc]{rights}}. A Dublin Core element, e.g.
  {\element[ns-elt=dc]{creator}} that is missing in lower {\element{metadata}} declaration
  (i.e. there is no element of the same name) is inherited from the upper ones.  So in
  {\myfigref{dc-inherits}}, the two boxes are equivalent, since the metadata in theory
  {\snippet{th1}} and in definition {\snippet{d1}} is inherited from the main declaration
  in the top-level {\element{omdoc}} element.  If there is a metadata element of the same
  name present, the closer one takes precedence.

\begin{erratum}[reported-by=Michael Kohlhase,date=2009-08-11]{{\texttt{for}} attribute on
    {\texttt{definition}} should be of type {\texttt{NCNames}}}
\setbox0=\hbox{%
\begin{minipage}{5.1cm}\footnotesize
\begin{lstlisting}[label=inheritancea,index={metadata,dc:creator},frame=none,numbers=none]
<omdoc xml:id="o1">
 <metadata>
  <dc:creator>MiKo</dc:creator>
 </metadata>
 
 <theory xml:id="th1">
 



  <symbol name="s1"/>
  <definition for="s1" xml:id="d1"/>




 </theory>

 <theory xml:id="th2">
  <metadata>
   <dc:creator>Paul</dc:creator>
  </metadata>
  <symbol name="s2"/>
  <definition for="s2" xml:id="d1">
   <metadata>
    <dc:creator>MiKo</dc:creator>
   </metadata>
  </definition>
 </theory>
</omdoc>
\end{lstlisting}
\end{minipage}}
\setbox1=\hbox{%
\begin{minipage}{5cm}\footnotesize
\begin{lstlisting}[label=inheritanceb,index={metadata,dc:creator},frame=none,numbers=none]
<omdoc xml:id="o1">
 <metadata>
  <dc:creator>MiKo</dc:creator>
 </metadata>
 
 <theory xml:id="th1">
  <metadata>
   <dc:creator>MiKo</dc:creator>
  </metadata>
  
  <symbol name="s1"/>
  <definition for="s1" xml:id="d1">
   <metadata>
    <dc:creator>MiKo</dc:creator>
   </metadata>
  </definition>
 </theory>

 <theory xml:id="th2">
  <metadata>
   <dc:creator>Paul</dc:creator>
  </metadata>
  <symbol name="s2"/>
  <definition for="s2" xml:id="d1">
   <metadata>
    <dc:creator>MiKo</dc:creator>
   </metadata>
  </definition>
 </theory>
</omdoc>
\end{lstlisting}
\end{minipage}}
\begin{myfig}{dc-inherits}{Inheritance of Metadata}
\fbox{\box0}$\hfill\longleftrightarrow\hfill$\fbox{\box1}
\end{myfig}
\end{erratum}

\end{tsection}
\end{tchapter}

%%% Local Variables: 
%%% mode: latex
%%% TeX-master: "omdoc"
%%% End: 

% LocalWords:  DCperson trl dateTime CC YY DD hh ss sss ISBN ISSN isbn IPR dc
% LocalWords:  MiKo aut clb edt ths trc lst sec qtmetadata lang mathescape Req
% LocalWords:  mtext camelcase natlist en fr nl creativecommons DRM
% LocalWords:  cctable comercial RDF CNX NNat mtxt ref xmlns inheritancea ns un
% LocalWords:  inheritanceb elt attr CMP Dataset omdoc metadata YYYY de eBook
% LocalWords:  creativecommons dc CC DRM cc cctable Req IANA RDF ccc lst xmlns
% LocalWords:  th
