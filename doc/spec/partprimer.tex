%%%%%%%%%%%%%%%%%%%%%%%%%%%%%%%%%%%%%%%%%%%%%%%%%%%%%%%%%%%%%%%%%%%%%%%%%
% This file is part of the LaTeX sources of the OMDoc 1.2 specifiation
% Copyright (c) 2006 Michael Kohlhase
% This work is licensed by the Creative Commons Share-Alike license
% see http://creativecommons.org/licenses/by-sa/2.5/ for details
\svnInfo $Id: partprimer.tex 8011 2008-09-07 19:43:48Z kohlhase $
\svnKeyword $HeadURL: https://svn.omdoc.org/repos/omdoc/branches/omdoc-1.2/doc/spec/partprimer.tex $
%%%%%%%%%%%%%%%%%%%%%%%%%%%%%%%%%%%%%%%%%%%%%%%%%%%%%%%%%%%%%%%%%%%%%%%%%

\part{An OMDoc Primer}\label{part:primer}
This part of the {\report} provides an easily approachable description of the {\omdoc}
format by way of paradigmatic examples of {\omdoc} documents.  The primer should be used
alongside the formal descriptions of the language contained in
{\mypartref{specification}}.

The intended audience for the primer are users who only need a casual exposure to the
format, or authors that have a specific text category in mind.  The examples presented
here also serve as specifications of ``best practice'', to give the readers an intuition
for how to encode various kinds of mathematical knowledge.

Each chapter of the {\omdoc} primer deals with a different category of mathematical
document and introduces new features of the {\omdoc} format in the context of concrete
examples.

\paragraph{\Mychapref{algebra}: Mathematical Textbooks and Articles} discusses the markup
process for an informal but rigorous mathematical texts.  We will use a fragment of
Bourbaki's ``Algebra'' as an example.  The development marks up the content in four steps,
from the document structure to a full formalization of the content that could be used by
automated theorem provers.  The first page of Bourbaki's ``Algebra'' serves as an example
of the treatment of a rigorous presentation of pure mathematics, as it can be found in
textbooks and articles.

\paragraph{\Mychapref{omcds} OpenMath Content Dictionaries} transforms an {\openmath}
content dictionary into an {\omdoc} document. {\openmath} content dictionaries are
semi-formal documents that serve as references for mathematical symbols in {\openmath}
encoded formulae.  As of {\openmath}2, {\omdoc} is an admissible {\openmath} content
dictionary format. They are a good example for mathematical glossaries, and background
references, both formal and informal.

\paragraph{\Mychapref{natlist} Structured and Parametrized Theories} shows the power of
theory markup in {\omdoc} for theory reuse and modular specification. The example builds a
theory of ordered lists of natural numbers from a generic theory of ordered lists and the
theory of natural numbers which acts as a parameter in the actualization process.

\paragraph{\Mychapref{dg-elal} A Development Graph for Elementary Algebra} extends the
range of theory-level structure by specifying the elementary algebraic hierarchy. The rich
fabric of relations between these theories is made explicit in the form of theory
morphisms, and put to use for proof reuse.

\paragraph{\Mychapref{courseware} Courseware and the Narrative/Content Distinction}
covers markup for a fragment of a computer science course in the {\omdoc} format, dwelling
on the difference between the narrative structure of the course and the background
knowledge. Course materials like slides or writings on blackboards are usually much more
informal than textbook presentations of mathematics. They also openly structure materials
by didactic criteria and leave out important parts of the rigorous development, which the
student is required to pick up from background materials like textbooks or the teacher's
recitation.

\paragraph{\Mychapref{rpc} Communication with and between Mathematical Software Systems}
uses an {\omdoc} fragment as content for communication protocols between mathematical
software systems on the Internet.  Since the communicating parties in this situation are
machines, {\omdoc} fragments are embedded into other {\xml} markup that serves as a
protocol for the distribution layer.

\medskip 

Together these examples cover many of the mathematical documents involved in communicating
mathematics. As the first two chapters build upon each other and introduce features of the
{\omdoc} format, they should be read in succession.  The remaining three chapters build on
these, but are largely independent.

To keep the presentation of the examples readable, we will only present salient parts of
the {\omdoc} representations in the discussion. The full text of the examples can be
accessed at \url{https://svn.omdoc/repos/omdoc/doc/spec/examples/spec}.

%%% Local Variables: 
%%% mode: latex
%%% TeX-master: "omdoc"
%%% End: 

% LocalWords:  OMDoc omdoc
% LocalWords:  provers omdoc formulae omcds natlist xmlrpc rpc dg elal
