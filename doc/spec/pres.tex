%%%%%%%%%%%%%%%%%%%%%%%%%%%%%%%%%%%%%%%%%%%%%%%%%%%%%%%%%%%%%%%%%%%%%%%%%
% This file is part of the LaTeX sources of the OMDoc 1.2 specifiation
% Copyright (c) 2006 Michael Kohlhase
% This work is licensed by the Creative Commons Share-Alike license
% see http://creativecommons.org/licenses/by-sa/2.5/ for details
\svnInfo $Id: pres.tex 8379 2009-06-11 05:32:28Z kohlhase $
\svnKeyword $HeadURL: https://svn.omdoc.org/repos/omdoc/branches/omdoc-1.2/doc/spec/pres.tex $
%%%%%%%%%%%%%%%%%%%%%%%%%%%%%%%%%%%%%%%%%%%%%%%%%%%%%%%%%%%%%%%%%%%%%%%%%

\begin{tchapter}[id=pres,short=Notation and Presentation]{Notation and Presentation (Module {\PRESmodule{spec}})}

As we have seen, {\omdoc} is concerned mainly with the content and structure of
mathematical documents, and offers a complex infrastructure for dealing with that.
However, mathematical texts often carry typographic conventions that cannot be determined
by general principles alone. Moreover, non-standard presentations of fragments of
mathematical texts sometimes carry meanings that do not correspond to the mathematical
content or structure proper. In order to accommodate this, {\omdoc} provides a limited
functionality for embedding style information into the document.
\begin{erratum}[date=2007-09-10,reported-by=Kristina Sojakova]{added CMP* to content of {\tt presentation} element}
\begin{myfig}{qtpres}{The {\omdoc} Elements for Notation Information}
\begin{scriptsize}
\begin{tabular}{|>{\tt}l|>{\tt}l|>{\tt}p{4.7truecm}|>{\tt}p{2truecm}|}\hline
{\rm Element}& \multicolumn{2}{l|}{Attributes\hspace*{2.25cm}} & Content  \\\hline
             & {\rm Required}  & {\rm Optional}     &           \\\hline\hline
 omstyle    & element & for, xml:id, xref, class, style & (style|xslt)* \\\hline
 presentation & for   & xml:id, xref, fixity, role, lbrack, rbrack, separator, 
                        bracket-style, class, style, precedence, crossref-symbol
                                           & CMP*, (use | xslt | style)* \\\hline
 xslt       & format  & xml:lang, requires, xref & {\xslt} fragment\\\hline 
 use        & format    
            & xml:lang, requires, fixity, precedence
              lbrack, rbrack, separator, element, 
              attributes, crossref-symbol         & (element | text | recurse | map | value-of)*\\\hline
\end{tabular}
\end{scriptsize}
\end{myfig}
\end{erratum}

The normal (but of course not the only) way to generate presentation from {\xml} documents
is to use {\xslt} style sheets (see {\mychapref{transform-xsl}} for other applications).
{\xslt}~\cite{Clark:xslt99} is a general transformation language for {\xml}.  {\xslt}
programs (often called {\defin{style sheet}s}) consist of a set of {\defin{template}s}
(rules for the transformation of certain nodes in the {\xml} tree). These templates are
recursively applied to the input tree to produce the desired output.

The general approach to presentation and notation in {\omdoc} is not to provide
general-purpose presentational primitives that can be sprinkled over the document, since
that would distract the author from the mathematical content, but to support the
specification of general style information for {\omdoc} elements and mathematical symbols
in separate elements.

In the case of a single {\omdoc} document it is possible to write a specialized
style sheet that transforms the content-oriented markup used in the document into
mathematical notation. However, if we have to deal with a large collection of
{\omdoc} representations, then we can either write a specialized style sheet for
each document (this is clearly infeasible to do by hand), or we can develop a
style sheet for the whole collection (such style sheets tend to get large and
unmanageable).

The {\omdoc} format allows to generate specialized style sheets that are tailored to the
presentation of (collections of) {\omdoc} documents. The mechanism will be discussed in
{\mychapref{transform-xsl}}, here we only concern ourselves with the {\omdoc} primitives
for representing the necessary data. In the next section, we will address the
specification of style information for {\omdoc} elements by {\element{omstyle}} elements,
and then the question of the specification of notation for mathematical symbols in
{\element{presentation}} elements.

\begin{tsection}[id=omstyle,short=Styling OMDoc Elements]{Specifying Style Information for {\omdoc} Elements}
  
  {\omdoc} provides the {\eldef{omstyle}}\footnote{This element would perhaps be more
    aptly be named {\snippet{omclass}}, since its function is more similar to the {\css}
    class concept\twin{CSS}{class}, but we keep the name {\element{omstyle}} for backwards
    compatibility in {\omdoc} 1.2.}  elements for specifying
  {\twintoo{style}{information}} for {\omdoc} elements.  An {\element{omstyle}} element
  has the attributes
  \begin{description}
  \item[{\attribute{element}{omstyle}}] This required attribute specifies the
    {\omdoc} element this style information should be applied to. The value of
    this attribute must be the full {\twintoo{qualified}{name}} (i.e.  including
    the {\indextoo{namespace}}) of the element.
  \item[{\attribute{for}{omstyle, presentation}}] This optional attribute allows
    to further restrict the {\omdoc} element to a single instance. The value of
    this attribute is a {\twintoo{URI}{reference}} to a single element.
  \item[{\attribute{xref}{omstyle, presentation, use, xslt, style}}] This optional
    attribute can be used to refer to another existing {\element{omstyle}} element (in
    another document via a {\twintoo{URI}{reference}}), sometimes avoiding double
    specification: If an {\element{omstyle}} element carries an {\attribute{xref}{omstyle,
        presentation}} attribute, its attributes and content is disregarded, and those of
    the target {\element{omstyle}} element is considered instead.
  \item[{\attribute{class}{omstyle, presentation}}] This optional attribute is an
    additional parameter that controls the output style. Remember that all {\omdoc}
    elements that have {\attributeshort[ns-attr=xml]{id}} attributes also carry a
    {\attributeshort{class}} attribute, which allows to specify different notational
    conventions (see {\mysecref{common-attribs}}): In the presentation of an {\omdoc}
    element only those {\element{omstyle}} elements are taken into account that have the
    same value in the {\attribute{class}{omstyle, presentation}} attribute.
    
    Note that the choice of notational style is not a content-carrying feature,
    and should not be depended on, indeed the value of the
    {\attribute{class}{presentation}} need not be respected by output routines,
    but can be overwritten.
\end{description}
In the presentation process described in {\mysecref{omdoc2pres}} the information specified
in the body of this element is then used to generate {\xslt} templates that are included
then into the generated style sheets. This information is either given directly in {\xslt}
using the {\eldef{xslt}} element, or in a {\eldef{style}} element using an
{\omdoc}-internal equivalent of a small subset of {\xslt}. The latter is used if the full
power of {\xslt} is not needed, and has the advantage that it can be transformed into the
input of other formatting engines.  The {\element{xslt}} and {\element{style}} elements
share the following attributes:
\begin{description}
\item[{\attribute{format}{use, xslt, style}}] This required attribute specifies the output
  format. Its value is a set of format specifiers divided by the {\snippet{|}} character.
  We use the specifiers {\attval{TeX}{format}{use}} for {\TeX} and {\LaTeX},
  {\attval{pmml}{format}{use}} for {\pmathml}, {\attval{cmml}{format}{use}} for
  {\cmathml}, {\attval{html}{format}{use}} for {\html},
  {\attval{mathematica}{format}{use}} for {\mathematica}
  notebooks\twin{notebook}{Mathematica}. Other formats can be specified at liberty.
  Finally, there is the pseudo format-specifier {\attval{default}{format}{use}}, which
  will be taken, if no other format is defined.  Note that case matters in these
  specifiers, so {\snippet{TeX}} is not the same as {\snippet{tex}}. Furthermore,
  {\attval{default}{format}{use}} is not a regular format specifier, so it cannot appear
  in the disjunctions.
\item[{\attribute[ns-attr=xml]{lang}{use, xslt, style}}] This specifies the languages for
  which this notation is used. Note that it is used differently than e.g. in the
  {\element{CMP}} element: on {\element{omstyle}}, the attribute
  {\attribute[ns-attr=xml]{lang}{use, xslt, style}} contains a whitespace-separated list
  of language specifiers and it does not have a default value {\snippet{en}}, if the
  attribute is not present, this means that this element is not language-specific.
\item[{\attribute{requires}{use, xslt, style}}] This attribute contains a
  {\twintoo{URI}{reference}} that points to a {\element{code}} element that contains a
  code fragment that is needed to be included for the presentation engine. For instance,
  the body of the {\element{omstyle}} element may contain {\TeX} macros that need to be
  defined.  Their definitions would need to be included in the output document by the
  presentation style sheet before they can be used.
\end{description}

{\mylstref{phrase}} shows a very simple example, where a {\element{phrase}} element is used to
mark a text passage as ``important''. Its {\attribute{class}{phrase}} attribute is picked
up by the {\element{omstyle}} element to prompt special treatment in the output.

\begin{lstlisting}[label=lst:phrase,
  caption={Specifying Style Information with the {\element{phrase}}
    Element.},index={omstyle,style,xslt,element,text,recurse,}]
<CMP>
  I want to mark <phrase xml:id="w1" class="important">this important
  text</phrase> as special.<phrase class="linebreak"/>
</CMP>

<omstyle element="omdoc:phrase" class="important">
  <style format='html|pmml'><element name="em"><recurse/></element></style>
  <xslt format='TeX' xmlns:xsl="http://www.w3.org/1999/XSL/Transform">
    <xsl:text>\emph{</xsl:text>
      <xsl:apply-templates/>
    <xsl:text>}</xsl:text>
  </xslt>
</omstyle>

<omstyle element="omdoc:phrase" class="linebreak">
  <style format='html|pmml'><element name="br"/></style>
  <style format='TeX'><text>\par\noindent</text></style>
</omstyle>
\end{lstlisting}
\end{tsection}

\begin{tsection}[id=style]{A Restricted Style Language}
  
  Let us now have a closer look at the presentation-language used in {\element{style}}
  elements.  In the first {\element{omstyle}} element in {\mylstref{phrase}} we see that
  the content of an {\element{xslt}} element is an {\xslt} fragment.  Note that when
  referring to {\omdoc} elements, the {\xslt} must use the full
  {\twintoo{qualified}{name}} (i.e. including the {\indextoo{namespace}}) of the elements
  for the presentation to work.\footnote{For DTD validation the {\xslt} fragments must be
    encoded using the {\tt{xsl:}} namespace prefix, unless the DTD has been adapted to a
    different prefix by setting the appropriate parameter entity.}
\begin{myfig}{qtstyle}{The {\omdoc} Elements for Styling}
\begin{scriptsize}
\begin{tabular}{|>{\tt}l|>{\tt}l|>{\tt}p{2.4truecm}|>{\tt}p{4.2truecm}|}\hline
{\rm Element}& \multicolumn{2}{l|}{Attributes\hspace*{2.25cm}} & Content  \\\hline
             & {\rm Required}  & {\rm Optional}     &           \\\hline\hline
 style      & format  & xml:lang, requires, xref & (element | text | recurse | map | value-of)*\\\hline 
 element    & name    & crid, cr, ns       & (attribute | element | text | 
                                              value-of  | recurse | map)*\\\hline
 attribute  & name    &                    & (value-of | text)*\\\hline
 text       &         &                    & (\#PCDATA)\\\hline
 value-of   & select  &                    & EMPTY\\\hline
 recurse    &         & select             & EMPTY\\\hline
 map        &         & select             & separator?, (element | text | recurse | map)\\\hline
 separator  &         &                    & (element | text | recurse | map)\\\hline
\end{tabular}
\end{scriptsize}
\end{myfig}
  
Let us analyze the example to see the presentation in action before we define it. In the
first {\element{style}} element in the {\element{omstyle}} for {\snippet{linebreak}} in
{\mylstref{phrase}} we see that the {\element{element}} element can be used to insert an
{\xml} element into the output; in this case it is the empty {\html} element
{\snippet{<br/>}}. In the second {\element{style}} child the {\element{text}} element (it
does not have attributes) allows to add arbitrary text into the output (in this case some
{\TeX} macros). In the first {\element{omstyle}} element, we see that the
{\element{element}} element may be non-empty, it contains the element {\element{recurse}},
which corresponds to the directive to continue presentation generation recursively over
the children of the element specified in the dominating {\element{omstyle}} element. The
effect of this is that the content of the first {\element{phrase}} element is encased in
the {\html} {\snippet{em}} element.

Textual material can be added to the output in two ways: by copying it from the source, or
supplying it in the transformation. For the latter, {\omdoc} supplies the {\eldef{text}}
element (it does not have attributes), which allows to add arbitrary text (its body) into
the output. For the former, we have the {\eldef{value-of}} element, an empty element that
carries the required attribute {\attribute{select}{value-of}}, whose value is an {\xpath}
expression. It adds the value (a string) to the {\xml} node specified by the expression to
the output.

The {\eldef{element}} element allows to generate {\xml} elements. It has a required
attribute {\attribute{name}{element}}, which contains its (local) name, and the optional
attribute {\attribute{ns}{element}} to specify the namespace. Attributes of the resulting
element can be specified by the {\eldef{attribute}} element: any {\element{attribute}}
element adds an {\twintoo{attribute-value}{pair}} of the form
{\snippet{\llquote{name}="\llquote{value}"}} to the output element specified by the
enclosing {\element{element}} element, where the local part {\llquote{name}} is the value
of the {\attribute{name}{attribute}} attribute (its namespace URI given by the value the
optional {\attribute{ns}{attribute}} attribute), and {\llquote{value}} is either the
result of presentation on the content of the {\element{attribute}} element or (iff that is
empty), the value of the {\xpath} expression in the optional
{\attribute{select}{attribute}} attribute.

To navigate the {\omdoc} structure to be transformed, we have two elements: the
{\element{recurse}} allows to specify a fragment continues presentation on a sub-element,
and the {\element{map}} element that maps directives over a set of sub-elements.  The
{\eldef{recurse}} element is empty, and can have the attribute
{\attribute{select}{recurse}}, which contains an {\xpath}~\cite{ClaDeR:xpath99} expression
specifying a set of {\omdoc} elements the presentation should continue with recursively.
If this attribute is missing, presentation continues on the children as in
{\mylstref{phrase}}. The {\eldef{map}} element (see~\mylstref{presentation} for an
example) has the optional attribute {\attribute{select}{map}} and contains a combination
of the transformation directive elements {\element{element}}, {\element{text}},
{\element{recurse}}, {\element{map}} after an optional {\element{separator}} child. The
{\element{map}} element directs the presentation engine to map the body
directives\footnote{i.e. those elements after the {\element{separator}} element} over the
list of elements specified by the {\xpath} expression in the {\attribute{select}{map}},
between any two elements, the result of styling the body of the {\element{separator}}
element is inserted between the result node sets. In {\mylstref{presentation}} the
{\element{map}} element recursively styles the children of the
{\element[ns-elt=om]{OMBVAR}} element and separates them by commata.  Furthermore, the
{\element{map}} element can have the attributes {\attribute{precedence}{map}},
{\attribute{lbrack}{map}}, and {\attribute{rbrack}{map}} to specify brackets (with
precedence-based elision) around the result. This is useful for generating argument
groups.

  Note that this {\omdoc}-internalized subset of {\xslt} restricts the expressivity of the
  presentation style by leaving out the computational features of {\xslt}.  Firstly, the
  infrastructure for iteration, recursion, variable declaration, \ldots is not present,
  and secondly, path expressions are restricted to pure {\xpath}~\cite{ClaDeR:xpath99},
  leaving out all {\xslt} extensions (e.g. functions calls), again leaving us with a more
  declarative subset of {\xslt}.
\end{tsection}

\begin{tsection}[id=presentation,short=Notation of Symbols]{Specifying the Notation of Symbols}
  In this section we discuss the problem of specifying the notation of mathematical
  symbols in {\omdoc}. The approach taken is very similar to the one for {\omdoc} elements
  presented in the previous section. The mathematical concepts and symbols introduced in
  an {\omdoc} document (by {\element{symbol}} elements or implicitly by abstract data
  types) often carry typographic conventions that cannot be determined by general
  principles alone. Therefore, these need to be specified, so that pleasing presentations
  can be generated.
  
  We have already seen the use of {\element{style}} and {\element{xslt}} elements for
  specifying the presentation of general {\omdoc} elements in the last section. Here we
  will present yet another way to specify presentation information that is specialized to
  notations of mathematical symbols.  The main idea is to specify the properties of
  mathematical symbols in relation to the representations of their children and siblings.
  
\begin{tsubsection}[id=pres-templates]{Specifying Notation via Templates}
  Let us build up our intuition by an example: For the notation information for the
  universal quantifier we would use an {\xslt} template like the one shown in
  {\mylstref{template}}.
\begin{lstlisting}[label=lst:template,language=XSLT,
  caption={An {\xslt} Template for the Universal Quantifier},mathescape,
  index={xsl:template,xsl:text,xsl:for-each,xsl:apply-templates,xsl:if}]
<xsl:template match="OMBIND[OMS[position()=1 and @name='forall' and @cd='quant1']]">
  <xsl:text>$\forall$</xsl:text>
  <xsl:for-each select="OMBVAR"/>
   <xsl:apply-templates/>
   <xsl:if test="position()!=last()">,</xsl:if>
  </xsl:for-each>
  <xsl:text>.</xsl:text>
   <xsl:apply-templates select="*[3]"/>
</xsl:template>
\end{lstlisting}
  The {\xpath} expression in the {\attributeshort{match}} attribute (the
  {\twindef{template}{head}}) specifies that this template acts as a presentation rule for
  {\element[ns-elt=om]{OMBIND}} elements, where the first child is of the form
  {\snippet{<OMS cd="quant1" name="forall"/>}}. Applied to such a node, the body of the
  template will be executed: it will print the quantifier $\forall$, then the bound
  variables as a comma-separated list (for each of the children of
  {\element[ns-elt=om]{OMBVAR}} it recursively applies {\xslt} templates from the style
  sheet), print a dot, and then recurse on the third child of the
  {\element[ns-elt=om]{OMBIND}} element.  Thus this template will print the {\openmath}
  expression below as $\allcdot{P,Q}{P\vee Q\Rightarrow Q\vee P}$ assuming appropriate
  templates for implication and disjunction.

\begin{lstlisting}[label=lst:or-comm,index={OMBIND,OMS,OMBVAR,OMV,OMA}]
<OMBIND>
  <OMS cd="quant1" name="forall"/>
  <OMBVAR><OMV name="P"/><OMV name="Q"/></OMBVAR>
  <OMA>
    <OMS cd="logic1" name="implies"/>
    <OMA><OMS cd="logic1" name="or"/>
      <OMV name="P"/>
      <OMV name="Q"/>
    </OMA>
    <OMA><OMS cd="logic1" name="or"/>
      <OMV name="Q"/>
      <OMV name="P"/>
    </OMA>
  </OMA>
</OMBIND>
\end{lstlisting}

  To annotate a symbol with notation information {\omdoc} supplies the
  {\eldef{presentation}} element. It is a top-level element whose
  {\attribute{for}{presentation}} attribute points to the symbol in question. It contains
  a multilingual {\element{CMP}} group that allows to specify the notation\footnote{Of
    course in the content markup in {\omdoc}, this looks somewhat awkward, since the
    representation relies on the fact that it will be rendered in the correct way. In the
    source, the whole markup looks somewhat circular.}.  Like the {\element{omstyle}}
  element, it has children that specify the presentation: The {\element{xslt}} element can
  be used to literally include the body of the template, and the {\element{style}} can
  express the presentation directives natively in {\omdoc}. In {\mylstref{presentation}}
  we have juxtaposed the presentational content from {\mylstref{template}} in
  {\element{xslt}} and {\element{style}} elements. Note that the directives in their body
  share much of the structure; the directives in the {\element{style}} are somewhat more
  succinct. The main difference to the {\xslt} template in {\mylstref{template}} is the
  specification of the template head: the attributes in the {\element{presentation}}
  element carry all the information necessary to identify the application conditions.

\begin{lstlisting}[label=lst:presentation,language=XSLT,
  caption={A Simple {\element{presentation}} Element for the Universal Quantifier},mathescape,
  index={xsl:template,xsl:text,xsl:for-each,xsl:apply-templates,xsl:if}]
<presentation for="#quant1.forall" role="binding">
  <CMP>We write 
    <OMOBJ>
      <OMBIND><OMS cd="quant1" name="forall"/>
        <OMBVAR><OMV name="X></OMBVAR>
        <OMV name="A"/>
      </OMBIND>
    </OMOBJ>
    for the phrase "A holds for all X". 
  </CMP> 
  <xslt format="default" xmlns:xsl="http://www.w3.org/1999/XSL/Transform">
    <xsl:text>$\forall$</xsl:text>
    <xsl:for-each select="OMBVAR"/>
     <xsl:apply-templates/>
     <xsl:if test="position()!=last()">,</xsl:if>
    </xsl:for-each>
    <xsl:text>.</xsl:text>
    <xsl:apply-templates select="*[3]"/>
  </xslt>
  <style format="html">
    <text>&#8704;</text>
    <map select="OMBVAR/*">
      <separator><text>,</text></separator>
      <recurse/>
    </map>
    <text>.</text>
    <recurse select="*[3]"/>
  </style>
  <style format="pmml">
    <element crid="." name="mrow" ns="http://www.w3.org/1998/Math/MathML">
      <element crid="*[1]" cr="yes" name="mo"><text>&#8704;</text></element>
      <element name="mrow" crid="*[2]">
        <map select="OMBVAR/*">
          <separator>
            <element name="mo" cr="yes">
              <attribute name="separator"><text>true</text></attribute>
              <text>,</text>
            </element>
          </separator>
          <recurse/>
        </map>
      </element>
      <recurse select="*[3]"/>
    </element>
  </style>
</presentation>
\end{lstlisting}

  The {\element{element}} element can have the {\attribute{crid}{element}} attribute which
  specifies the role of the generated element in parallel markup of mathematical formulae
  (see~\mysubsecref{math-markup:mathml}). The value of this element (if present) must be a
  {\xpath} fragment (see~\cite{ClaDeR:xpath99}) pointing to the element in the source that
  semantically corresponds to the generated element (see
  {\mylstref{presentation}}\footnote{There the top-level generated {\snippet{mrow}}
    element corresponds to the application as specified by the path ``{\snippet{.}}'',
    whereas its first child corresponds to the quantifier symbol, and the bound variables
    correspond to each other.}). Finally, the {\element{element}} element can carry the
  {\attribute{cr}{element}} attribute, which (if its value is {\attval{yes}{cr}{element}})
  instructs the presentation system to to set an {\snippet{xlink:href}} attribute on the
  result element that acts as a {\indextoo{cross-reference}} to the symbol declaration.

\end{tsubsection}


\begin{tsubsection}[id=pres-declarative]{Specifying Notation via Syntactic Roles}
  Note that hand-coding {\xslt}-templates is a tedious and error-prone process, and that
  we need a template for each output format (e.g. {\LaTeX}, {\html}, {\pmathml}, ASCII),
  and even various output languages (for instance the greatest common divisor of two
  integers is expressed by the symbol $gcd$ in English but $ggT$ (``gr\"o{\ss}ter
  gemeinsamer Teiler'') in German). Obviously, the respective templates for all of these
  transformations share a great deal of structure (in our example, they only differ in the
  representation of the glyph for the quantifier itself). 

  Therefore {\omdoc} goes another step and supplies a set of abbreviations that are
  sufficient for most presentation applications via the {\element{use}} elements that can
  occur as children of {\element{presentation}} elements. The user only needs to specify
  the relevant information in the {\element{use}} elements and a separate translation
  process generates the needed {\xslt} templates from that (see
  {\mychapref{transform-xsl}}). The {\eldef{use}} elements make use of the same symbolic
  attributes and specialize (over-define) these attributes according to the respective
  format and language.  The following set of attributes are particular to the
  {\element{presentation}}, since they are independent of the language and the output
  format.

\begin{description}
\item[{\attribute{for}{presentation}}, {\attribute{xref}{presentation}},
  {\attribute{class}{presentation}}] (see the specification for {\element{omstyle}} in the
  last section)
\item[{\attribute{role}{presentation}}] This attribute specifies to which roles of
  the symbol the {\element{presentation}} element applies. The value of this
  attribute can be one of
  \begin{description}
  \item[{\attval{applied}{role}{presentation}}] for situations, where the symbol
    occurs as a function symbol that is applied to a list of arguments, i.e. as
    the first child of an {\element[ns-elt=om]{OMA}} or an {\element[ns-elt=m]{apply}} element.
  \item[{\attval{binding}{role}{presentation}}] for situations, where the symbol
    occurs as a binding symbol, i.e as the first child of an {\element[ns-elt=om]{OMBIND}}
    element or an {\element[ns-elt=m]{apply}} element that is followed by an
    {\element[ns-elt=m]{bvar}} element.
  \item[{\attval{key}{role}{presentation}}] for situations, where the symbol occurs as a
    key in an attribution, i.e. as a child of an {\element[ns-elt=om]{OMATTR}} element at
    an odd position ({\cmathml} does not have the attribution construct).
  \end{description}
In the examples in {\myfigref{function-style}} we have assumed the head to be an
{\element[ns-elt=om]{OMA}} element (for functional application). It can also be an
{\element[ns-elt=om]{OMBIND}} as in the case of a quantifier in
{\myfigref{ombind-presentation}}.
\item[{\attribute{fixity}{presentation, use}}] This optional attribute can be one of the
  keywords {\attval{prefix}{fixity}{presentation}} (the default),
  {\attval{infix}{fixity}{presentation}}, {\attval{postfix}{fixity}{presentation}}, and
  {\attval{assoc}{fixity}{presentation}}. The value {\attval{assoc}{fixity}{presentation}}
  has two variants: {\attval{infixl}{fixity}{presentation}} and
  {\attval{infixr}{fixity}{presentation}}, which have the same presentation;
  {\attval{infixl}{fixity}{presentation}} is used for a binary infix operator that
  associates to the left like the list constructor in Standard ML,
  {\attval{infixr}{fixity}{presentation}} is the right-leaning analogon. 

  If the {\attribute{fixity}{presentation, use}} attribute is given, then it determines
  the placement of the symbol specified in the {\attribute{for}{presentation}}
  attribute. For {\attval{prefix}{fixity}{presentation}} it is placed in front of the
  arguments, (this is the generic mathematical function notation). For
  {\attval{postfix}{fixity}{presentation}} the function is put behind the arguments, e.g.
  for derivatives: $f'$. The case {\attval{infix}{fixity}{presentation}} is reserved for
  binary operators, where the function is inserted between the two arguments. Finally,
  {\attval{assoc}{fixity}{presentation}} is used for associative operators like addition,
  it puts the function symbol between any two arguments.
  
  Note that {\attval{infix}{fixity}{presentation}} is almost a special case of
  {\attval{assoc}{fixity}{presentation}}, but since it is reserved for binary
  operators, it disregards any arguments but the first two.
\item[{\attribute{bracket-style}{presentation}}] The
  {\attribute{fixity}{presentation, use}} information can be combined with the
  bracketing style, which can be either
  {\attval{lisp}{bracket-style}{presentation}} ({\snippetin{LISP}}-style brackets) or
  {\attval{math}{bracket-style}{presentation}} (generic mathematical function
  notation which is the default).
  
  {\Myfigref{function-style}} shows some combinations of attributes and their
  results on the function style.
\item[{\attribute{precedence}{presentation}}] allows us to specify the operator precedence
  in order to elide unnecessary brackets. The {\omdoc} presentation system orients itself
  on the {\scsys{Prolog}} standard: lower precedences mean stronger binding, and brackets
  can be omitted. If we set the default precedence to 1000, and other precedences as
  specified in {\myfigref{precedence}}, then the formulae below are presented as $(x+2)^2$
  and $x+y^2$, respectively.
  \begin{center}
\begin{lstlisting}[numbers=none,index={OMA,OMS,OMV}]
 <OMA>                                     <OMA>
   <OMS cd="arith1" name="power"/>           <OMS cd="arith1" name="plus"/>   
   <OMA>                                     <OMV name="x"/>
     <OMS cd="arith1" name="plus"/>          <OMA>
     <OMV name="x"/>                           <OMS cd="arith1" name="power"/>
     <OMV name="y"/>                           <OMV name="y"/>
   </OMA>                                      <OMI>2</OMI>
   <OMI>2</OMI>                              </OMA>
 </OMA>                                    </OMA>
\end{lstlisting}
\end{center}
\begin{myfig}{precedence}{Common Operator Precedences}
  \begin{tabular}{|l|l|l|}\hline
    Precedence & Operators                  & Comment\\\hline\hline
    200    & +,-                        & unary \\\hline
    200    & $\hat{}$                   & exponentiation \\\hline
    400    & $*,\land,\cap$             & multiplicative \\\hline
    500    & $+,-,\lor,\cup$            & additive\\\hline
    600    & /                          & fraction \\\hline
    700    & $=, \ne, \leq, <, >, \geq$ & relation\\\hline
  \end{tabular}
\end{myfig}
\end{description}
The next set of attributes can occur both in {\element{presentation}} and
{\element{use}} elements. If they occur in both, then the values of those
specified on the {\element{use}} elements take precedence over those specified in
the dominating {\element{presentation}} element. 
\begin{description}
\item[{\attribute{lbrack}{presentation, use}}/{\attribute{rbrack}{presentation, use}}]
  These two attributes handle the brackets to be used in presentation of a complex
  expression.  They will be used unless elided according to the precedence.
\item[{\attribute{separator}{presentation, use}}] This specifies the separator in
  the argument list of a function. The default for {\attribute{separator}{presentation, use}} is
  the {\indextoo{comma}}.  See {\myfigref{function-style}} for some combinations.
  \begin{myfig}{function-style}{Attribute-Combination and Function Style}
  \begin{tabular}{|c|c|c||c|}\hline
     {\snippet{fixity}}  & {\snippet{bracket-style}}  & {\snippet{separator}}  & yields       \\\hline\hline
     {\snippet{prefix}}  & {\snippet{lisp}} & `` '' & $ (f\; 1\; 2\; 3)$ \\\hline
     {\snippet{postfix}} & {\snippet{lisp}} & `` '' & $(1\; 2\; 3\; f)$  \\\hline
     {\snippet{prefix}}  & {\snippet{math}} & ``{\snippet{,}}'' &$f(1,2,3)$  \\\hline
     {\snippet{postfix}} & {\snippet{math}} & ``{\snippet{,}}'' &$(1,2,3)f$ \\\hline\hline
     \multicolumn{4}{|c|}{assuming {\snippet{lbrack="("}} and {\snippet{rbrack=")"}}}\\\hline
  \end{tabular}
  \end{myfig}
\item[{\attribute{crossref-symbol}{presentation, use}}] This attribute specifies to which
  parts of the symbol's presentation {\indextoo{cross-reference}s} should be attached to:
  in some formats like {\html}, and recently also in {\LaTeX} (thanks to the
  {\snippetin{hyperref.sty}} package), it may be useful to attach a hyperlink from the
  presentation of the symbol to its definition.  Some symbols are constructed by using the
  {\attribute{lbrack}{presentation, use}} and {\attribute{rbrack}{presentation, use}}, or
  the {\attribute{separator}{presentation, use}} attributes as part of the symbol
  presentation. For instance, in the notation $(a,b)$ for pairs, the binary function
  symbol for pairing is really composed of three parts ``('', ``)'', and ``,'', which
  should all be cross-referenced. The attribute's values
  {\attval{no}{crossref-symbol}{presentation}},
  {\attval{yes}{crossref-symbol}{presentation}},
  {\attval{brackets}{crossref-symbol}{presentation}},
  {\attval{separator}{crossref-symbol}{presentation}},
  {\attval{lbrack}{crossref-symbol}{presentation}},
  {\attval{rbrack}{crossref-symbol}{presentation}}
  {\attval{all}{crossref-symbol}{presentation}} can be used to specify this behavior.
  {\attval{no}{crossref-symbol}{presentation}} means cross-referencing is forbidden,
  {\attval{yes}{crossref-symbol}{presentation}} -- which is the default value -- means
  cross-referencing only on the print-form of the function symbol,
  {\attval{lbrack}{crossref-symbol}{presentation}},
  {\attval{rbrack}{crossref-symbol}{presentation}},
  {\attval{brackets}{crossref-symbol}{presentation}}, only on the left/right/both
  brackets, {\attval{separator}{crossref-symbol}{presentation}}, on the separator, and
  finally {\attval{all}{crossref-symbol}{presentation}} on all presentation parts.
  
  In {\myfigref{ombind-presentation}}, the effect of the default
  {\attval{yes}{crossref-symbol}{presentation}} can be seen in the lower part of
  the figure: the {\LaTeX} and the {\html} presentations have attached hyperlinks
  to the representation of the universal quantifier.
\end{description}

\begin{myfig}{ombind-presentation}
{Notation for {\snippet{forall}} (cf. {\mylstref{template}}) using {\snippet{presentation}}}
\setbox0=\hbox{\begin{minipage}{5cm}
\begin{lstlisting}[frame=none,numbers=none,index={presentation,use}]
<presentation for="#forall" 
              role="binding"    
              separator=".">     
 <use format="TeX">\forall</use>
 <use format="html">&#8704;</use>
</presentation>                 
\end{lstlisting}
\end{minipage}}
\setbox1=\hbox{\begin{minipage}{5cm}
\begin{lstlisting}[frame=none,numbers=none,index={OMBIND,OMS,OMBVAR,OMV}]
<OMBIND>
  <OMS cd="quant1" name="forall"/>
  <OMBVAR>
    <OMV name="X"/>
  </OMBVAR>
  <OMS cd="logic1" name="true"/>
</OMBIND>
\end{lstlisting}
\end{minipage}}
\setbox2=\hbox{\footnotesize\begin{tabular}{ll}
      \LaTeX: & \verb+\href{../ocd/logic1.ps#true}{\forall}X.+\\
      & \verb+\href{../ocd/logic1.ps#true}{{\sf true}}+\\
      \html: & \snippet{<a href="../ocd/logic1.html\#forall">\&\#8704;</a> X.}\\
      &\snippet{<a href="../ocd/logic1.html\#true"><b>true</b></a>}
\end{tabular}}
\begin{tabular}{|c|c|}\hline
  Notation specification & Example\\\hline
  \box0 & \box1 \\\hline
 \multicolumn{2}{|p{11cm}|}{Using {\xslt} templates induced from the 
   {\element{presentation}} element on the {\openmath} expression yields
  $\allcdot{X}{\sf true}$, where the glyph  $\forall$ carries a
  hyperlink\footnote{given a 
  suitable output device like a browser or a recent version of {\snippetin{dvips}}} to
  it definition, as the {\attribute{crossref-symbol}{presentation}} on the
  {\element{presentation}} element has the default value
  {\attval{yes}{crossref-symbol}{presentation}}. Internally, the hyperlinks are
  format-dependent, we have: 
  \begin{center}\box2\vspace*{-2ex}\end{center}} \\\hline
\end{tabular}
\end{myfig}
The next set of attributes can only appear on the {\element{use}} attribute, since
they are only meaningful for selected output formats. 

\begin{description}
\item[{\attribute{format}{use, xslt, style}}, {\attribute[ns-attr=xml]{lang}{use, xslt,
      style}}, {\attribute{requires}{use, xslt, style}}] (see the specification
  for {\element{xslt}} and {\element{style}} above).
\item[{\attribute{element}{use}}, {\attribute{attributes}{use}},
  {\attribute{bracket-style}{use}}] These attributes simplify the specification of
  notations in {\xml}-based formats like {\mathml}\index{mathml@{\mathml}}. The
  {\attribute{element}{use}} attribute contains the name and the
  {\attribute{attributes}{use}} the attribute declarations of an {\xml} element that takes
  the place of the brackets specified in the attributes {\attribute{lbrack}{presentation,
      use}} and {\attribute{rbrack}{presentation, use}}. If the attribute
  {\attribute{fixity}{use}} is used on a {\element{use}} element in conjunction with the
  {\attribute{element}{use}} and {\attribute{attributes}{use}} attributes, then it
  specifies the position of the element brackets rather than the brackets specified in the
  {\attribute{lbrack}{presentation, use}} and {\attribute{rbrack}{presentation, use}}
  attributes.
    
  For instance, the {\twintoo{binomial}{coefficient}} is some presented as $\left({n\atop
      m}\right)$ (spoken ``$n$ choose $m$'') and represented as
\begin{lstlisting}
  <mfrac linethickness='0'><mi>n</mi><mi>m</mi></frac>
\end{lstlisting}
  in {\pmathml}. The first {\element{presentation}} element in {\mylstref{binomial}} shows
  a {\element{presentation}} element that has this effect. The second
  {\element{presentation}} element in {\mylstref{binomial}} shows a notation declaration,
  which applied to
\begin{lstlisting}
<OMA><OMS cd="arith" name="power"/>
  <OMI>3</OMI><OMI>5</OMI>
</OMA>
\end{lstlisting}
  would yield {\snippet{3<sup>5</sup>}} for the target {\snippet{html}}.
\end{description}


\begin{lstlisting}[label=lst:binomial,
  caption={Presentation for Binomial Coefficients},
  index={presentation,use}]
<presentation for="#binomial" role="applied">
  <use format="default" fixity="infix">choose</use> 
  <use format="TeX" lbrack="\bigl({" rbrack="}\bigr)">\atop</use>
  <use format="pmml" element="mfrac" attributes="linethickness='0'"/>
</presentation>

<presentation for="#power" role="applied" fixity="infix" 
   crossref-symbol="no" precedence="200" bracket-style="lisp">
  <use format="html" fixity="prefix" bracket-style="math" element="sup"/>
  <use format="TeX">^</use>
  <use format="pmml" element="msup" fixity="prefix"/>
</presentation>
\end{lstlisting}  

Conceptually, the attributes of the {\element{presentation}} and {\element{use}}
elements form a meta-language for {\xslt} style sheets that aims at covering the
most common notations succinctly and legibly. In situations, where this language
does not suffice, we must fall back to to {\element{style}} or even
{\element{xslt}} elements.
\end{tsubsection}
\end{tsection}

\begin{tsection}[id=pres-bound]{Presenting Bound Variables}
  
  As we have seen in {\mysecref{sem-var}}, the presentation approaches for symbols do not
  work for (bound) variables\twin{bound}{variable}\footnote{We say that an
    {\element[ns-elt=om]{OMBIND}} element binds\index{binding} a variable {\snippet{<OMV
        name="x"/>}}, iff this {\element[ns-elt=om]{OMBIND}} element is the nearest one,
    such that {\snippet{<OMV name="x"/>}} occurs in (second child of the
    {\element[ns-elt=om]{OMATTR}} element in) the {\element[ns-elt=om]{OMBVAR}} child
    (this is the {\twindef{defining}{occurrence}} of {\snippet{<OMV name="x"/>}}). For
    content {\mathml}, the definition is analogous, only that an
    {\element[ns-elt=m]{apply}} element with {\element[ns-elt=m]{bvar}} child takes the
    role of the {\element[ns-elt=om]{OMBIND}} and {\element[ns-elt=om]{OMBVAR}}
    elements.}, as there is no independent place to put the {\element{presentation}}
  element. In this section, we will present the {\omdoc} solution to this problem. The
  main idea is simply to annotate {\twintoo{defining}{occurrence}s} of variables with
  notation information. Without this, we are forced to use the ASCII variable name in
  {\openmath} and a translation of the {\pmathml} in the {\element[ns-elt=m]{ci}} element
  for other formats in {\mathml}. This is hardly adequate for modern mathematics, where
  variables are numbered, decorated with primes or change marks, and cast in other colors
  or font families for better recognition.
  
  In {\omdoc} we follow the spirit of the {\openmath} standard~\cite{BusCapCar:2oms04}
  which suggests to annotate (via {\element[ns-elt=om]{OMATTR}} parts of) the {\openmath}
  objects with notation information by {\element{presentation}} elements. Unlike
  {\openmath}, we restrict this practice to {\twintoo{defining}{occurrence}s} of
  {\twintoo{bound}{variable}s}, since all the other constructs can be handled with the
  methods introduced above.  We use the symbol {\snippet{<OMS cd="omdoc"
      name="notation"/>}} symbol to identify the following object as a notation
  declaration and the {\element[ns-elt=om]{OMFOREIGN}} element to hold it.

\begin{erratum}[reported-by=Alberto Gonzales Palomo,date=2006-10-06]{The {\texttt{for}}
    attribute should be {\texttt{\#X4}} instead of {\texttt{\#X}} in listings 19.5 and 19.6}
\begin{lstlisting}[language=OpenMath,label=lst:notation-om,
    caption={Notation for Bound Variables in {\openmath}},
    index={OMOBJ,OMBIND,OMS,OMBVAR,OMV,OMATTR,OMATP}]
<OMOBJ>                            
  <OMBIND>                          
    <OMS cd="quant1" name="forall"/> 
    <OMBVAR>                         
      <OMATTR>                        
        <OMATP>                        
          <OMS cd="omdoc" name="notation"/>
          <OMFOREIGN encoding="application/omdoc+xml">
            <presentation for="#X"> 
              <use format="TeX">X_4</use>
              <use format="pmml">
                <msub><mi>X</mi><mn>4</mn></msub>
              </use>
              <use format="html">X<sub>4</sub></use>
            </presentation>
          </OMFOREIGN>
        </OMATP>
        <OMV name="X4"/>                
      </OMATTR>                        
    </OMBVAR>
    <OMA><OMS cd="relation1" name="eq"/>
      <OMV name="X4"/>
      <OMV name="X4"/>
    </OMA>
  </OMBIND>                         
</OMOBJ>
\end{lstlisting}
  To represent binding objects in {\cmathml} we follow a very similar strategy, using the
  {\element[ns-elt=m]{semantics}} element to associate the defining occurrence of the
  bound variable with its notation declaration, which is embedded into the
  {\element[ns-elt=m]{annotation-xml}} child.

\begin{lstlisting}[language=MathML,label=lst:notation-mathml,
     caption={Notation for Bound Variables in {\cmathml}},
     index={math,apply,forall,bvar,ci,csymbol}]
<m:math>
  <m:apply>  
    <m:forall/>
    <m:bvar>
      <m:semantics>
        <m:ci><m:msub><m:mi>X</m:mi><m:mn>4</m:mn></m:msub></m:ci>
        <m:annotation-xml encoding="application/xml+OMDoc"
          definitionURL="http://omdoc.org/omdoc.omdoc#notation">
          <presentation for="#X4"> 
            <use format="TeX">X_4</use>
            <style format="pmml">
              <element name="msub" ns="http://www.w3.org/1998/Math/MathML">
                <element name="mi" ns="http://www.w3.org/1998/Math/MathML">
                  <text>X</text>
                </element>
                <element name="mn" ns="http://www.w3.org/1998/Math/MathML">
                  <text>4</text>
                </element>
              </element>
            </style>
            <style format="html">
              <text>X</text>
              <element name="sub" ns="http://www.w3.org/1999/xhtml">
                <text>4</text>
              </element>
            </style>
          </presentation>
        </m:annotation-xml>
      </m:semantics>
    </m:bvar>
    <m:apply><m:eq/><m:cn>4</m:cn><m:cn>4</m:cn></m:apply>
  </m:apply>
</m:math>
\end{lstlisting}
\end{erratum}
With these declarations, all the variables in the scope of the universal quantifier
  would be represented as $X_4$, yielding $\allcdot{X_4}{X_4=X_4}$ which is exactly what
  we wanted. Note that if we want to specify notations for function variables ({\omdoc}
  does not prevent the user from doing this), we need to also specify notations for the
  {\twintoo{non-applied}{occurrence}s} of the symbol --- otherwise a fallback using the
  variable name has to be used. For instance, to make the (false) conjecture that all
  relations are symmetric we could use the following representation:
\begin{lstlisting}[language=OpenMath,label=notation-om-function,
    caption={Notation for bound variables in {\openmath}},
    index={OMOBJ,OMBIND,OMS,OMBVAR,OMV,OMATTR,OMATP}]
<OMOBJ xmlns="http://www.openmath.org/OpenMath">
  <OMBIND>                          
    <OMS cd="quant1" name="forall"/> 
    <OMBVAR>                         
      <OMATTR>                        
        <OMATP>                        
          <OMS cd="omdoc" name="notation"/>
          <OMFOREIGN encoding="application/omdoc+xml">
            <presentation xmlns="http://www.mathweb.org/omdoc"
                          for="#R" role="applied" precedence="500" fixity="infix"> 
              <use format="TeX">\prec</use>
              <use format="pmml|html">&#x022DE;</use>
            </presentation>
            <presentation xmlns="http://www.mathweb.org/omdoc" for="#R"> 
              <use format="TeX">{}\prec{}</use>
              <use format="pmml|html">&#x022DE;</use>
            </presentation>
          </OMFOREIGN>
        </OMATP>
        <OMV name="R"/>                
      </OMATTR>
      <OMV name="X"/>
    </OMBVAR>
    <OMA><OMV name="R"/><OMV name="X"/><OMV name="X"/></OMA>
  </OMBIND>                         
</OMOBJ>
\end{lstlisting}
This would give us the presentation $\allcdot{{}\prec{},X}{X\prec X}$. Here, the first
occurrence of the variable ${}\prec{}$ is handled by the second notation declaration (it
does not occur in applied position), the second occurrence of $\prec$ is in applied
position, so the second notation declaration governs this and puts it in to infix
position. Note that while {\omdoc} allows to specify this kind of notation declarations,
they should be used with great care and discretion.  In this particular case, the infix
notation of $\prec$ de-emphasizes the variable nature, and might lead to confusion;
moreover, the particular choice of the glyph $\prec$ may suggest irreflexivity, which may
or may not be intended.
\end{tsection}
\end{tchapter}
%%% Local Variables: 
%%% mode: latex
%%% TeX-master: "omdoc"
%%% End: 

% LocalWords:  xsl omstyle xref  xslt pmml cmml mathematica tex lang en PCDATA
% LocalWords:  lst linebreak OMSTR em br href om omlet cd quant comm qtpres gr
% LocalWords:  ter gemeinsamer Teiler sen ta tion ombind arith lbrack rbrack mi
% LocalWords:  crossref hyperref sty dvips mathml mfrac linethickness omclass
% LocalWords:  frac lt gt msup pres mroot var bvar ci OMATP msub mn eq attribs
% LocalWords:  csymbol definitionURL cn mathescape sem OMFOREIGN qtstylen crid
% LocalWords:  cr ns mrow xlink rt mo xmlns attr elt infixl infixr CDATA
% LocalWords:  recurse omdoc html CMP qtstyle OMBVAR forall OMV OMA OMOBJ gcd
% LocalWords:  ggT OMATTR Prolog OMI
