%%%%%%%%%%%%%%%%%%%%%%%%%%%%%%%%%%%%%%%%%%%%%%%%%%%%%%%%%%%%%%%%%%%%%%%%%
% This file is part of the LaTeX sources of the OMDoc 1.2 specifiation
% Copyright (c) 2006 Michael Kohlhase
% This work is licensed by the Creative Commons Share-Alike license
% see http://creativecommons.org/licenses/by-sa/2.5/ for details
\svnInfo $Id: algebra.tex 8483 2009-08-11 08:16:04Z kohlhase $
\svnKeyword $HeadURL: https://svn.omdoc.org/repos/omdoc/branches/omdoc-1.2/doc/spec/algebra.tex $
%%%%%%%%%%%%%%%%%%%%%%%%%%%%%%%%%%%%%%%%%%%%%%%%%%%%%%%%%%%%%%%%%%%%%%%%%

\begin{tchapter}[id=algebra,short=Textbooks and Articles]{Mathematical Textbooks and Articles}

  In this chapter we will work an example of a stepwise formalization of mathematical
  knowledge. This is the task of e.g. an editor of a mathematical textbook preparing it
  for web-based publication.  We will use an informal, but rigorous text: a fragment of
  Bourbaki's Algebra~\cite{Bourbaki:a74}, which we show in {\myfigref{bourbaki}}. We will
  mark it up in four stages, discussing the relevant {\omdoc} elements and the design
  decisions in the {\omdoc} format as we go along.  Even though the text was actually
  written prior to the availability of the {\TeX/\LaTeX} system, we will take a {\LaTeX}
  representation as the starting point of our markup experiment, since this is the
  prevalent source markup format in mathematics nowadays.
\setbox1=\hbox{\cite{Bourbaki:a74}}
\begin{myfig}{bourbaki}{A fragment from Bourbaki's algebra~\usebox1}
\hspace*{-12pt}\fbox{\begin{minipage}{11cm}\small
      \noindent{\bf 1. LAWS OF COMPOSITION}\vspace{1em}\par\noindent
      {\sc Definition 1.} {\emph{Let $E$ be a set. A mapping of $E\times E$ is
        called a law of composition on $E$. The value $f(x,y)$ of $f$ for an
        ordered pair $(x,y)\in E\times E$ is called the composition of $x$ and $y$
        under this law. A set with a law of composition is called a magma.}}
      \vspace{1em}

      The composition of $x$ and $y$ is usually denoted by writing $x$ and $y$ in a
      definite order and separating them by a characteristic symbol of the law in question
      (a symbol which it may be agreed to omit). Among the symbols most often used are $+$
      and $\cdot$, the usual convention being to omit the latter if desired; with these
      symbols the composition of $x$ and $y$ is written respectively as $x+y$, $x.y$ or
      $xy$. A law denoted by the symbol $+$ is usually called {\emph{addition}} (the
      composition $x+y$ being called the {\emph{sum}} of $x$ and $y$) and we say that it
      is {\emph{written additively}}; a law denoted by the symbol $.$ is usually called
      {\emph{multiplication}} (the composition $x.y=xy$ being called the {\emph{product}}
      for $x$ and $y$) and we say that it is {\emph{written multiplicatively}}.
      
      In the general arguments of paragraphs 1 to 3 of this chapter we shall generally use
      the symbols $\top$ and $\bot$ to denote arbitrary laws of composition.
      
      By an abuse of language, a mapping of a {\emph{subset}} of $E\times E$ into $E$ is
      sometimes called a law of composition {\emph{not everywhere defined}} on $E$.
      \vspace{1em}

      \strut\hfill
      \begin{minipage}{11cm}\parindent=1.5em
        {\emph{Examples}}. (1) The mappings $(X,Y)\mapsto X\cup Y$ and $(X,Y)\mapsto
        X\cap Y$ are laws of composition on the set of subsets of a set $E$.
        
        (2) On the set $\bf N$ of natural numbers addition, multiplication, and
        exponentiation are laws of composition (the compositions of $x\in{\bf N}$
        and $y\in{\bf N}$ under these laws being denoted respectively by $x+y$,
        $xy$, or $x.y$ and $x^y$) ({\em{Set Theory}}, III, $\S3$, no. 4).
  
        (3) Let $E$ be a set; the mapping $(X,Y)\mapsto X\circ Y$ is a law of
        composition on the set of subsets of $E\times E$ ({\emph{Set Theory}}, II,
        $\S3$, no. 3, Definition 6); the mapping $(f,g)\mapsto f\circ g$ is a law
        of composition on the set of mappings from $E$ into $E$ ({\emph{Set Theory}},
        II, $\S5$, no. 2).
      \end{minipage}
    \end{minipage}}
\end{myfig}

{\Mysecref{minimal-omdoc}} discusses the minimal markup that is needed to turn an
arbitrary document into a valid {\omdoc} document --- albeit one, where the markup is
worthless of course. It discusses the necessary {\xml} infrastructure and adds some
meta-data to be used e.g. for {\twintoo{document}{retrieval}} or
archiving\twin{document}{archiving} purposes.

In {\mysecref{top-level}} we mark up the top-level structure of the text and
classify the paragraphs by their category as mathematical statements.  This level
of markup already allows us to annotate and extract some meta-data and would allow
applications to slice the text into individual units, store it in databases like
{\mbase} (see {\mysecref{mbase}}), or the In2Math knowledge
base~\cite{Dahn:sbt01,BauBlo:adtmpd01}, or assemble the text slices into
individualized books e.g. covering only a sub-topic of the original work.
However, all of the text itself, still contains the {\LaTeX} markup for formulae,
which is readable only by experienced humans, and is fixed in notation.  Based on
the segmentation and meta-data, suitable systems like the {\activemath} system
described in {\mysecref{activemath}} can re-assemble the text in different orders.
   
In {\mysecref{formulae}}, we will map all mathematical objects in the text into
{\openmath} or {\cmathml} objects. To do this, we have to decide which
symbols we want to use for marking up the formulae, and how to structure the
theories involved. This will not only give us the ability to generate specialized
and user-adaptive notation for them (see {\mychapref{transform-xsl}}), but also to
copy and paste them to symbolic math software systems. Furthermore, an assembly
into texts can now be guided by the semantic theory structure, not only by the
mathematical text categories or meta-data.
   
Finally, in {\mysecref{formalization}} we will fully formalize the mathematical
knowledge. This involves a transformation of the
{\twintoo{mathematical}{vernacular}} in the statements into some logical
formalism. The main benefit of this is that we can verify the mathematical
contents in theorem proving environments like {\nuprl}~\cite{Constable86},
{\hol}~\cite{GoMe93}, {\mizar}~\cite{Rudnicki:aomp92} and
OMEGA~\cite{BenzmuellerEtAl:otama97}.

\begin{tsection}[id=minimal-omdoc]{Minimal OMDoc Markup}

  It actually takes very little change to an existing document to make it a valid {\omdoc}
  document. We only need to wrap the text into the appropriate {\xml} document tags. In
  {\mylstref{outerpart}}, we have done this and also added
  {\indextoo{meta-data}}. Actually, since the {\element{metadata}} and the document type
  declaration are optional in {\omdoc}, just wrapping the original text with lines 1, 4,
  7, 31, 32, and 36 to 38 is the simplest way to create an {\omdoc}
  document.\medskip
\begin{lstlisting}[label=lst:outerpart,mathescape,
    caption={The outer part of the document},
    index={DOCTYPE,omdoc,metadata,dc:title,dc:creator,dc:date,dc:description,dc:source,
           dc:type,dc:format,theory,omtext,CMP}]
<?xml version="1.0" encoding="utf-8"?>
<!DOCTYPE omdoc PUBLIC "-//OMDoc//DTD OMDoc Basic V1.2//EN"
                "http://omdoc.org/dtd/omdoc-basic.dtd" []>

<omdoc xml:id="algebra1.omdoc" version="1.2" modules="@basic"
       xmlns:dc="http://purl.org/dc/elements/1.1/" 
       xmlns:cc="http://creativecommons.org/ns"
       xmlns="http://www.mathweb.org/omdoc">
  <metadata>
    <dc:title>Laws of Composition</dc:title>
    <dc:creator role="trl">Michael Kohlhase</dc:creator> 
    <dc:date action="created">2002-01-03T07:03:00</dc:date>
    <dc:date action="updated">2002-11-23T18:17:00</dc:date>
    <dc:description>
      A first migration step for a fragment of Bourbaki's Algebra
    </dc:description>
    <dc:source>
      Nicolas Bourbaki, Algebra, Springer Verlag 1989, ISBN 0-387-19373-1
    </dc:source>
    <dc:type>Text</dc:type>
    <dc:format>application/omdoc+xml</dc:format>
    <dc:rights>Copyright (c) 2005 Michael Kohlhase</dc:rights>
    <cc:license>
      <cc:permissions reproduction="permitted" distribution="permitted" 
                      derivative_works="permitted"/>
      <cc:prohibitions commercial_use="permitted"/>
      <cc:requirements notice="required" copyleft="required" attribution="required"/>
    </cc:license>
  </metadata>

  <omtext xml:id="all">
    <CMP xml:lang="en">
      {\sc Definition 1.} Let $E$ be a set. A mapping $E\times E$ is called a law of
      $\ldots$
      mappings from $E$ into $E$ ({\emph{Set Theory}}, II, $\S5$, no. 2).
    </CMP>
  </omtext>
</omdoc>
\end{lstlisting}\medskip

\noindent \renewcommand{\baselinestretch}{.97}We will now explain
the general features of the {\omdoc} representation in detail by
line numbers. The references point to the relevant sections in the
{\omdoc} specification; details and normative rules for using the
elements in questions can be found there.%\pagebreak

We will now explain the general features of the {\omdoc} representation in detail
by line numbers. The references point to the relevant sections in the {\omdoc}
specification; details and normative rules for using the elements in questions can
be found there.
\begin{small}
\begin{longtable}{|l|p{8.6cm}|p{.8cm}|}\hline
  line & Description & ref.\\\hline\hline
1 & This document is an {\xml} 1.0 file
  that is encoded in the {\twintoo{UTF-8}{encoding}}. 
  & \\\hline
2,3 & The parser is told to use a document type
       definition for validation. The string {\snippet{omdoc}} specifies the name of
       the root element, the identifier {\snippet{PUBLIC}} specifies that the DTD (we use
      the ``{\omdoc} basic'' DTD; see {\mysubsecref{sub-languages:basic}}),
      which can be identified by the {\twintoo{public}{identifier}} in the first string
      and looked up in an {\xml} catalog\twin{XML}{catalog} or (if that fails) can be 
      found at the URL specified in the second string. 

      A DTD declaration is not strictly needed for an {\omdoc} document, but is
      recommended, since the DTD supplies default values  for some attributes. 
  & {\ref{sec:validate-dtd}}  p.~\pageref{sec:validate-dtd}\\\hline
4 & In general, {\xml} files can contain as much whitespace as they want between elements,
    here we have used it for structuring the document. & \\\hline
5 & Start tag\twin{start}{tag} of the root element
    of the document. It declares the version ({\omdocv{1.2}}) via the
    {\attribute{version}{omdoc}}, and an identifier of the document using the
    {\attribute[ns-attr=xml]{id}{omdoc}} attribute. The optional {\attribute{modules}{omdoc}}
    specifies the sub-language used in this document. This is used when no DTD is present
    (see {\mysubsecref{sub-languages:basic}}).
  & {\ref{eldef:omdoc}}  p.~\pageref{eldef:omdoc}\\\hline
6,7 & the {\atwintoo{namespace}{prefix}{declaration}s} for the Dublin Core\twin{Dublin
    Core}{namespace}, Creative Commons\twin{Creative Commons}{namespace}, and
    {\openmath}\twin{OpenMath}{namespace} namespaces. They declare the prefixes
    {\snippetin{dc:}}, {\snippetin{cc:}}, and {\snippetin{om:}}, and bind them to the
    specified {\indextoo{URI}s}. We will need the {\openmath} namespace only in the
    third markup step described in {\mysecref{formulae}}, but spurious namespace prefix
    declarations are not a problem in  the {\xml} world. 
  & {\ref{chap:spec-intro}}  p.~\pageref{chap:spec-intro}\\\hline
8 & the {\twintoo{namespace}{declaration}} for the document; if not prefixed, all
    elements live in the {\omdoc} namespace\twin{OMDoc}{namespace}.
  & {\ref{sec:omdoc-ns}}  p.~\pageref{sec:omdoc-ns}\\\hline
9--29 & The {\indextoo{metadata}} for the whole document in Dublin Core
  format 
  & {\ref{eldef:metadata}}  p.~\pageref{eldef:metadata}\\\hline
10 & The title of the  document &\ref{eldef:dc:title} p.~\pageref{eldef:dc:title} \\\hline
11 & The document creator, here in the role of a {\indextoo{translator}}
  & {\ref{sec:dc-roles}}  p.~\pageref{sec:dc-roles}\\\hline
12 & The date and time of first creation of the document in {\atwintoo{ISO}{8601}{norm}}
    format.  & {\ref{eldef:dc:date}} p.~\pageref{eldef:dc:date} \\\hline
13 & The date and time of the last update to the document in {\atwintoo{ISO}{8601}{norm}}
    format. & {\ref{eldef:dc:date}} p.~\pageref{eldef:dc:date} \\\hline
14--16 & A short description of the contents of the document
   & {\ref{eldef:dc:description}} p.~\pageref{eldef:dc:description} \\\hline
17--19 & Here we acknowledge that the {\omdoc} document is just a translation from an
     earlier work. & {\ref{eldef:dc:source}} p.~\pageref{eldef:dc:source} \\\hline
20 & The type of the document, this can be {\snippetin{Dataset}} (un-ordered 
     mathematical knowledge) or {\snippetin{Text}} (arranged for human consumption). 
     & {\ref{eldef:dc:type}} p.~\pageref{eldef:dc:type} \\\hline
21 & The format/{\twintoo{MIME}{type}}~\cite{FreBor:MIME96} of the document, for {\omdoc}, this is
     {\snippetin{application/omdoc+xml}}. 
     & {\ref{eldef:dc:format}} p.~\pageref{eldef:dc:format} \\\hline
22 & The copyright resides with the creator of the {\omdoc} document
     & {\ref{eldef:dc:rights}} p.~\pageref{eldef:dc:rights} \\\hline
23--28 & The creator licenses the document to the world under certain conditions
     as specified in the {\twintoo{Creative Commons}{license}} specified in this element.
     & {\ref{eldef:cc:license}} p.~\pageref{eldef:cc:license} \\\hline
24,25 & The {\element[ns-elt=cc]{permissions}} element gives the world the {\indextoo{permission}} to
     reproduce\index{reproduction} and distribute\index{distribution} it
     freely. Furthermore the license grants the public the right to make derivative works
     under certain conditions.
     & {\ref{eldef:cc:permissions}} p.~\pageref{eldef:cc:permissions} \\\hline
26 & The {\element[ns-elt=cc]{prohibitions}} can be used to prohibit certain uses of the document,
     but this one is unencumbered.
     & {\ref{eldef:cc:prohibitions}} p.~\pageref{eldef:cc:prohibitions} \\\hline
27 & The {\element[ns-elt=cc]{requirements}} states conditions under which the license is
     granted. In our case the licensee is required to keep the copyright notice and
     license notices intact during distribution, to give credit to the copyright holder,
     and that any derivative works derived from this document must be licensed under the
     same terms as this document (the {\twintoo{copyleft}{clause}}).
     & {\ref{eldef:cc:requirements}} p.~\pageref{eldef:cc:requirements} \\\hline
31-37 & The {\element{omtext}} element  is used to mark up text fragments. Here, we
     have simply used a single {\element{omtext}} to classify the  whole text in the
     fragment as unspecific ``text''. 
   & {\ref{eldef:omtext}} p.~\pageref{eldef:omtext} \\\hline
32-36 & The {\element{CMP}} element holds the actual text in a
     {\twintoo{multilingual}{group}}. Its {\attribute[ns-attr=xml]{lang}{CMP}} specifies the language. 
     If the document is used with a {\indextoo{DTD}} or an {\xml} schema\twin{XML}{schema} (as we are)
     this attribute is redundant, since the default value given by the DTD or schema
     is {\attvalshort[ns-attr=xml]{en}{lang}}. More keywords in other languages can be given
     by adding more {\element{CMP}} elements.  
    & {\ref{eldef:CMP}} p.~\pageref{eldef:CMP} \\\hline
33--35 & The text of the {\LaTeX} fragment we are migrating. For simplicity we do
    not change the text, and leave that to later stages of the migration. & \\\hline
38 & The {\twintoo{closing}{tag}} of the root {\element{omdoc}}
     element. There may not be text after this in the file. 
   & {\ref{eldef:omdoc}} p.~\pageref{eldef:omdoc} \\\hline
\end{longtable}
\end{small}
\end{tsection}

\begin{tsection}[id=top-level,short=Structure and Statements]{Marking up the text
    structure and statements}
  
  In the next step, we analyze and mark up the structure of the text of the further, and
  embed the paragraphs into markup for mathematical statements or text segments.  Instead
  of lines \erratumReplace[reported-by=Krzysztof Retel,date=2006-09-14]{wrong reference}{19--25}{32--36} in
  {\mylstref{outerpart}}, we will now have the representation in
  {\mylstref{segmented-text}}.

\begin{lstlisting}[label=lst:segmented-text,mathescape,frame=topline,
    caption={The segmented text},
    index={theory,symbol,definition,omtext,CMP,omgroup,metadata,dc:title,example}]
<omtext xml:id="magma.def" type="definition">
  <CMP>Let <legacy format="TeX">$E$</legacy> be a set $\ldots$ called a magma.</CMP>
</omtext>

<omtext xml:id="t1">
  <CMP>The composition of <legacy format="TeX">$x$</legacy> $\ldots$  multiplicatively.</CMP>
</omtext>
<omtext xml:id="t2">
  <CMP>In the general $\ldots$ composition.</CMP>
</omtext>
<omtext xml:id="t3">
  <CMP>By an abuse $\ldots$ on <legacy format="TeX">$E$.</legacy></CMP>
</omtext> 

<omgroup xml:id="magma-ex" type="enumeration">
   <metadata><dc:title>Examples</dc:title></metadata>

   <omtext type="example" xml:id="e1.magma">
     <CMP>
       The mappings <legacy format="TeX">$(X,Y)$</legacy> 
       $\ldots$ subsets of a set <legacy format="TeX">$E$</legacy>.
     </CMP>
   </omtext>
   <omtext type="example" xml:id="e2.magma">
     <CMP>
       On the set <legacy format="TeX">$\bf N$</legacy> $\ldots$ III, $\S3$, no. 4).
     </CMP>
   </omtext>
   <omtext type="example" xml:id="e3.magma">
     <CMP>
       Let <legacy format="TeX">$E$</legacy> be a set; $\ldots$ II, $\S5$, no. 2).
     </CMP>
   </omtext>
</omgroup>
\end{lstlisting}

In summary, we have sliced\index{slicing} the text into {\element{omtext}} fragments and
individually classified them by their mathematical role. The formulae inside have been
encapsulated into {\element{legacy}} elements that specify their format for further
processing. The {\twintoo{higher-level}{structure}} has been captured in {\omdoc} grouping
elements and the document as well as some of the slices have been annotated by metadata.
\begin{small}
\begin{erratum}[reported-by=Krzysztof Retel,date=2006-09-14]{wrong cross-reference for ``line 16''}
\begin{longtable}{|l|p{8.6cm}|p{.8cm}|}\hline
  line & Description & ref.\\\hline\hline
1 & The {\element{omtext}} element classifies the text fragment as  a
    {\attval{definition}{type}{omtext}}, other types for mathematical statements include 
    {\attval{axiom}{type}{omtext}}, {\attval{example}{type}{omtext}},
    {\attval{theorem}{type}{omtext}}, and {\attval{lemma}{type}{omtext}}. Note that the
    numbering of the original text is lost, but can be re-created in the text presentation
    process. The optional {\attribute[ns-attr=xml]{id}{omtext}} attribute specifies a
    document-unique identifier that can be used for reference later. 
    &\ref{eldef:omtext} p.~\pageref{eldef:omtext}\\\hline
2 & A {\twintoo{multilingual}{group}} of {\element{CMP}} elements that hold the  text (in our
    case, there is only the English default). Here
    the {\TeX} formulae have been marked up with {\element{legacy}} elements
    characterizing them as such. This might simplify a later
    automatic transformation to {\openmath} or {\cmathml}.
  & {\ref{eldef:legacy}} p.~\pageref{eldef:legacy} \\\hline
4--13 & We have classified every paragraph in the original as a separate
   {\element{omtext}} element, which does not carry a {\attribute{type}{omtext}} since it
   does not fit any other mathematical category at the moment. 
   & {\ref{eldef:omtext}} p.~\pageref{eldef:omtext} \\\hline
15 & The three examples in the original in {\myfigref{bourbaki}} are grouped into
   an enumeration. We use the {\omdoc}  {\element{omgroup}} element for this. The
   optional attribute {\attribute[ns-attr=xml]{id}{omgroup}} can be used for referencing later. We have chosen
   {\attval{enumeration}{type}{omgroup}} for the {\attribute{type}{omgroup}}
   attribute to specify the numbering of the examples in the original.
   & {\ref{eldef:omgroup}}  p.~\pageref{eldef:omgroup} \\\hline
16 & We can use the {\element{metadata}} of the {\element{omgroup}} element to
   accommodate the title ``Examples'' in the original. We could enter more metadata
   at this level.   & {\ref{eldef:dc:title}} p.~\pageref{eldef:dc:title} \\\hline
18 & The {\attribute{type}{omtext}} attribute of this {\element{omtext}} element
     classifies this text fragment as an example.
   & {\ref{eldef:omtext}} p.~\pageref{eldef:omtext} \\\hline
 \end{longtable}
\end{erratum}
 \end{small}
\end{tsection}

\begin{tsection}[id=formulae]{Marking up the Formulae}
 
  After we have marked up the top-level structure of the text to expose the content, the
  next step will be to mark up the formulae in the text to content mathematical form. Up
  to now, the formulae were still in {\TeX} notation, which can be read by {\TeX}/{\LaTeX}
  for presentation to the human user, but not used by symbolic mathematics software. For
  this purpose, we will re-represent the formulae as {\openmath} objects or {\cmathml},
  making their functional structure\twin{abstract}{syntax}\twin{structure}{abstract}
  explicit.

  So let us start turning the {\TeX} formulae in the text into {\openmath} objects. Here
  we use the hypothetical {\url{mbc.mathweb.org}} as repository for theory collections.

\begin{erratum}[reported-by=Michael Kohlhase,date=2009-08-11]{{\texttt{for}} attribute on
    {\texttt{definition}} should be of type {\texttt{NCNames}}}
\begin{erratum}[reported-by=Krzysztof Retel,date=2006-09-14]{should be "definiendum" not "definiens"}
\begin{erratum}[reported-by=Gordan Ristovski,date=2008-07-11]{should be definiendum-applied not definiens-applied}
\begin{lstlisting}[label=lst:formulae-definition,mathescape,
    caption={The definition of a magma with {\openmath} objects},
    index={definition,CMP,om:OMOBJ,om:OMS,om:OMA,om:OMV}]
<!DOCTYPE omdoc PUBLIC "-//OMDoc//DTD OMDoc CD V1.2//EN" 
                       "http://omdoc.org/dtd/omdoc-cd.dtd" 
          [<!ENTITY % om.prefixed "INCLUDE">]>

<theory xml:id="magmas">
  <imports from="background.omdoc#products"/>
  <imports from="http://mbc.mathweb.org/omstd/relation1.omdoc#relation1"/>

  <symbol name="magma">
    <metadata><dc:description>Magma</dc:description></metadata>
  </symbol>
  <symbol name="law_of_composition"/>

  <definition xml:id="magma.def" for="magma law_of_composition">
    <CMP> 
      Let <om:OMOBJ><om:OMV name="E"/></om:OMOBJ> be a set. A mapping of 
      <om:OMOBJ>
        <om:OMA><om:OMS cd="products" name="Cartesian-product"/>
          <om:OMV name="E"/><om:OMV name="E"/>
        </om:OMA>
      </om:OMOBJ> is called a 
      <term cd="magmas" name="magma" role="definiendum">law of composition</term>
      on <om:OMOBJ><om:OMV name="E"/></om:OMOBJ>.  The value 
      <om:OMOBJ>
        <om:OMA><om:OMV name="f"/>
          <om:OMV name="x"/><om:OMV name="y"/>
        </om:OMA>
      </om:OMOBJ>
      of <om:OMOBJ><om:OMV  name="f"/></om:OMOBJ> for an ordered pair
      <om:OMOBJ>
        <om:OMA><om:OMS cd="sets" name="in"/>
          <om:OMA><om:OMS cd="products" name="pair"/>
            <om:OMV name="x"/><om:OMV name="y"/>
          </om:OMA>
          <om:OMA><om:OMS cd="products" name="Cartesian-product"/>
            <om:OMV name="E"/><om:OMV name="E"/>
          </om:OMA>
        </om:OMA>
      </om:OMOBJ> is called the 
      <term cd="magmas" name="law_of_composition" 
                        role="definiendum-applied">composition</term>
       of <om:OMOBJ><om:OMV name="x"/></om:OMOBJ> and 
      <om:OMOBJ><om:OMV name="y"/></om:OMOBJ> under this law. 
      A set with a law of composition is called a 
      <term cd="magmas" name="magma" role="definiendum">magma</term>.
    </CMP>
  </definition>
$\ldots$
</theory>
$\ldots$
\end{lstlisting}
\end{erratum}
\end{erratum}
\end{erratum}
Of course all the other mathematical statements in the documents have to be treated in the
same way.
\begin{small}
\begin{longtable}{|l|p{8.6cm}|p{.8cm}|}\hline
  line & Description & ref.\\\hline\hline 
  1--4 & The {\snippetin{omdoc-basic}} {\twintoo{document type}{definition}} is no longer
      sufficient for our purposes, since we introduce new symbols that can be used in
      other documents. The  {\indextoo{DTD}} for {\omdoc} content
      dictionaries\twin{content dictionary}{OMDoc} (see {\mychapref{omcds}}),
      which allows this. Correspondingly, we would specify the value
      {\attval{cd}{module}{omdoc}} for the attribute {\attribute{module}{omdoc}}.

      The part in line 4 is the {\twintoo{internal}{subset}} of the {\indextoo{DTD}},
      which sets a {\twintoo{parameter}{entity}} for the modularized DTD to instruct it to
      accept {\openmath} elements in their {\twintoo{namespace}{prefixed}} form. Of course
      a suitable {\twintoo{namespace prefix}{declaration}} is needed as well. 
     & {\ref{subsec:sub-languages:cd}} p.~\pageref{subsec:sub-languages:cd} \\\hline
  5   & The {\twintoo{start}{tag}} of a theory. We need this, since symbols and
       definitions  can only appear inside {\element{theory}} elements.  
       & {\ref{sec:theories}} p.~\pageref{sec:theories}\\\hline 
  6,7 & We need to import the theory {\snippet{products}} to  be able to use symbols
          from it in the definition below. The  value of the {\attribute{from}{imports}}
          is a relative {\twintoo{URI}{reference}} to a {\element{theory}} element much
          like the one in line  5. The other
          {\element{imports}} element imports the theory {\snippet{relation1}} from the
          {\openmath} standard content dictionaries\index{content dictionary}\footnote{The
            originals are available at 
            \url{http://www.openmath.org/cd}; see {\mychapref{omcds}} for a discussion
            of the differences of the original {\openmath} format and the {\omdoc}
            format used here.}.  Note that we do not need to import the theory
          {\snippet{sets}} here, since this is already imported by
      the theory {\snippet{products}}. 
    & {\ref{eldef:imports}} p.~\pageref{eldef:imports}\\\hline 
9--11 & A symbol declaration: For every definition, {\omdoc} requires the declaration of
        one or more {\element{symbol}} elements for the concept that is to be defined. The
        {\attribute{name}{symbol}} attribute is used to identify 
        it.  The {\element[ns-elt=dc]{description}} element allows to supply a multilingual (via the
        {\attribute[ns-attr=xml]{lang}{description}} attribute) group of keywords for the
        declared symbol 
     & {\ref{eldef:symbol}} p.~\pageref{eldef:symbol}\\\hline 
  12 & Upon closer inspection it turns out that the definition in
       {\mylstref{formulae-definition}} actually defines three concepts: ``law of
       composition'', ``composition'', and ``magma''. Note that ``composition'' is just
       another name for the value under the law of composition, therefore we do not need
       to declare a symbol for this. Thus we only declare one for ``law of composition''.
     & {\ref{eldef:symbol}} p.~\pageref{eldef:symbol}\\\hline 
  14 & A definition: the {\element{definition}} element carries a
       {\attribute{name}{definition}} attribute for reference within the theory. We need
       to reference the two symbols defined here in the {\attribute{for}{definition}} attribute 
       of the  {\element{definition}} element; it takes a whitespace-separated list of
      {\attribute{name}{symbol}} attributes of {\element{symbol}} elements in the same 
      theory as values.  
    & {\ref{eldef:definition}} p.\pageref{eldef:definition} \\\hline 
  16 & We use an {\openmath} object for the set $E$. It is an
      {\element[ns-elt=om]{OMOBJ}} element with an {\element[ns-elt=om]{OMV}} daughter, whose
      {\attribute{name}{OMV}} attribute specifies the object to be a variable with
      name $E$. We have chosen to represent the set $E$ as a variable instead of a
      constant (via an {\element[ns-elt=om]{OMS}} element) in the theory, since it seems to be
      local to the definition. We will discuss this further in the next section, where
      we talk about formalization.  
    & {\ref{eldef:om:OMOBJ}} p.~\pageref{eldef:om:OMOBJ} \\\hline 
  17--21 & This {\element[ns-elt=om]{OMOBJ}} represents the {\twintoo{Cartesian}{product}}
          $E\times E$ of the set $E$ with itself.
      It is an application (via an {\element[ns-elt=om]{OMA}} element) of the symbol for the
      binary Cartesian product {\indextoo{relation}} to $E$.  
    & {\ref{eldef:om:OMA}} p.~\pageref{eldef:om:OMA} \\\hline 
  18 & The symbol for the Cartesian product
       {\indextoo{constructor}} is represented as an {\element[ns-elt=om]{OMS}} element. The
       {\attribute{cd}{OMS}} attribute specifies the theory that defines the symbol,
       and the {\attribute{name}{OMS}} points to the {\element{symbol}} element in it
       that declares this symbol. The value of the {\attribute{cd}{OMS}} attribute is a
       theory identifier. Note that this theory has to be imported into the
       current theory, to be legally used.
     & {\ref{eldef:om:OMS}} p.~\pageref{eldef:om:OMS}\\\hline 
   22 & We use the {\element{term}} element to characterize the defined terms in the text
        of the definition.  Its {\attribute{role}{term}} attribute can used to mark the
        text fragment as a {\attval{definiens}{role}{term}}, i.e. a concept that is under
        definition.   
      & {\ref{eldef:term}} p.~\pageref{eldef:term}\\\hline 
24--28 & This object stands   for $f(x,y)$ &  \\\hline 
30--39 & This object represents $(x,y)\in E\times E$.
         Note that we make use of the symbol for the elementhood relation from the
         {\openmath} core {\indextoo{content dictionary}} {\snippet{set1}} and of the
         {\indextoo{pair}{constructor}} from the theory of
         products from the Bourbaki collection there.  & \\\hline
\end{longtable}
\end{small}
\noindent The rest of the representation in
{\mylstref{formulae-definition}} is analogous. Thus we have treated
the first definition in {\myfigref{bourbaki}}. The next two
paragraphs contain notation conventions that help the human reader
to understand the text. They are annotated as {\element{omtext}}
elements. The third paragraph is really a definition (even if the
wording is a bit bashful), so we mark it up as one in the style of
{\mylstref{formulae-definition}} above.

Finally, we come to the examples at the end of our fragment. In the markup shown
in {\mylstref{example-formulae}} we have decided to construct a new theory for
these examples since the examples use concepts and symbols that are independent of
the theory of magmas. Otherwise, we would have to add the {\element{imports}}
element to the theory in {\mylstref{formulae-definition}}, which would have
mis-represented the actual dependencies. Note that the new theory has to import
the theory {\snippet{magmas}} together with the theories from which examples are taken,
so their symbols can be used in the examples. 
\begin{lstlisting}[label=lst:example-formulae,mathescape,
    caption={Examples for magmas with {\openmath} objects},
    index={example,CMP,OMOBJ,OMBIND,OMS,OMA,OMV}]
<theory xml:id="magmas-examples">
  <metadata><dc:title>Examples</dc:title></metadata>

   <imports from="http://mbc.mathweb.org/omstd/fns1.omdoc##fns1"/>
   <imports from="background.omdoc#nat"/>
   <imports from="background.omdoc#functions"/>
   <imports from="#magmas"/>
   
  <omgroup xml:id="magma-ex" type="enumeration">
    <metadata><dc:title>Examples</dc:title></metadata>

    <example xml:id="e1.magma" for="law_of_composition" type="for">
      <CMP>The mappings 
        <om:OMOBJ>
          <om:OMBIND><om:OMS cd="fns1" name="lambda"/>
            <om:OMBVAR>
              <om:OMV name="X"/><om:OMV name="Y"/>
            </om:OMBVAR>
            <om:OMA><om:OMS cd="functions" name="pattern-defined"/>
              <om:OMA><om:OMS cd="products" name="pair"/>
                <om:OMV name="X"/>
                <om:OMV name="Y"/>
              </om:OMA>
              <om:OMA><om:OMS cd="sets" name="union"/>
                <om:OMV name="X"/>
                <om:OMV name="Y"/>
              </om:OMA>
            </om:OMA>
          </om:OMBIND>
        </om:OMOBJ> and 
        <om:OMOBJ>
          <om:OMBIND><om:OMS cd="fns1" name="lambda"/>
            <om:OMBVAR>
              <om:OMV name="X"/><om:OMV name="Y"/>
            </om:OMBVAR>  
            <om:OMA><om:OMS cd="functions" name="pattern-defined"/>
              <om:OMA><om:OMS cd="products" name="pair"/>
                <om:OMV name="X"/>
                <om:OMV name="Y"/>
              </om:OMA>
              <om:OMA><om:OMS cd="sets" name="intersection"/>
                <om:OMV name="X"/>
                <om:OMV name="Y"/>
              </om:OMA>
            </om:OMA>
          </om:OMBIND>
        </om:OMOBJ>
        are <term cd="magmas" name="law_of_composition>laws of composition</term>
        on the set of subsets of a set 
        <om:OMOBJ><om:OMS cd="magmas" name="E"/></om:OMOBJ>.
      </CMP>
    </example>
    
    <example xml:id="e2.magma" for="law_of_composition" type="for">
      <CMP>
        On the set <om:OMOBJ><om:OMS cd="nat" name="Nat"/></om:OMOBJ> 
        of <term cd="nats" name="nats">natural numbers</term>, 
        <term cd="nats" name="plus">addition</term>, 
        <term cd="nats" name="times">multiplication</term>, and 
        <term cd="nats" name="power">exponentiation</term> are $\ldots$
      </CMP>
    </example>
  </omgroup>
</theory>
\end{lstlisting}

\noindent The {\element{example}} element in line 13 is used for
mathematical examples of a special form in {\omdoc}: objects that
have or fail to have a specific property. In our case, the two given
mappings have the property of being a law of composition. This
structural property is made explicit by the
{\attribute{for}{example}} attribute that points to the concept that
these examples illustrate, in this case, the symbol
{\snippet{law\_of\_composition}}. The {\attribute{type}{example}}
attribute has the values {\attval{for}{type}{example}} and
{\attval{against}{type}{example}}.  In our case
{\attval{for}{type}{example}} applies,
{\attval{against}{type}{example}} would for counterexamples. The
content of an {\element{example}} is a multilingual {\element{CMP}}
group. For examples of other kinds --- e.g. usage examples, {\omdoc}
does not supply specific markup, so we have to fall back to using an
{\element{omtext}} element with type
{\attval{example}{type}{omtext}} as above.

In our text fragment, where the examples are at the end of the section that deals
with magmas, creating an independent theory for the examples (or even multiple
theories, if examples from different fields are involved) seems appropriate. In
other cases, where examples are integrated into the text, we can equivalently embed
theories into other theories. Then we would have the following structure:

\begin{lstlisting}[label=lst:example-embedded,mathescape,
    caption={Examples embedded into a theory},
    index={theory,imports}]
<theory xml:id="magmas">
  <imports xml:id="imp3" from="background.omdoc#products"/>
  <imports from="http://mbc.mathweb.org/omstd/relation1.omdoc#relation1"/>
  $\ldots$
  <theory xml:id="magmas-examples"
    <imports xml:id="imp4" 
       from="http://omdoc.org/examples/omstd/fns1.omdoc#fns1"/>
    <imports xml:id="imp5" from="background.omdoc#nat"/>
    <imports xml:id="imp6" from="background.omdoc#functions"/>
    $\ldots$
  </theory>
  $\ldots$
</theory>
\end{lstlisting}
Note that the embedded theory ({\snippet{magmas-examples}}) has access to all the
symbols in the embedding theory ({\snippet{magmas}}), so it does not have to import it.
However, the symbols imported into the embedded theory are only visible in it, and do
not get imported into the embedding theory.
\end{tsection}

\begin{tsection}[id=formalization]{Full Formalization}
      
  The final step in the migration of the text fragment involves a transformation of the
  {\twintoo{mathematical}{vernacular}} in the statements into some logical formalism. The
  main benefit of this is that we can verify the mathematical contents in theorem proving
  environments.  We will start out by dividing the first definition into two parts. The
  first one defines the symbol {\snippet{law\_of\_composition}}
  (see~\mylstref{law_of_composition-formal}), and the second one {\snippet{magma}}
  (see~\mylstref{magma:formal}).

\begin{erratum}[reported-by=Michael Kohlhase,date=2009-08-11]{{\texttt{for}} attribute on
    {\texttt{definition}} should be of type {\texttt{NCNames}}}
\begin{lstlisting}[label=lst:law_of_composition-formal,
    caption={The formal definition of a law of composition},
    index={definition,CMP,OMOBJ,OMBIND,OMBVAR,OMS,OMA,OMV}]
<symbol name="law_of_composition">
  <metadata><dc:description>A law of composition on a set.</dc:description></metadata>
</symbol>
<definition xml:id="magma.def" for="law_of_composition" type="simple">
  <CMP> 
    Let <om:OMOBJ><om:OMV name="E"/></om:OMOBJ> be a set. A mapping of 
    <om:OMOBJ><om:OMR href="#comp.1"/></om:OMOBJ> 
    is called a <term cd="magmas" name="law_of_composition" 
                      role="definiens">law of composition</term>
    on <om:OMOBJ><om:OMV name="E"/></om:OMOBJ>.
  </CMP>
  <om:OMOBJ>
    <om:OMBIND>
      <om:OMS cd="fns1" name="lambda"/>
      <om:OMBVAR>
        <om:OMV name="E"/><om:OMV name="F"/>
      </om:OMBVAR>
      <om:OMA><om:OMS cd="pl0" name="and"/>
       <om:OMA><om:OMS cd="sets" name="set"/>
          <om:OMV name="E"/>
       </om:OMA>
       <om:OMA>
         <om:OMS cd="functions" name="function"/>
         <om:OMA id="comp.1">
           <om:OMS cd="products" name="Cartesian-product"/>
           <om:OMV name="E"/>
           <om:OMV name="E"/>
         </om:OMA>
         <om:OMV name="E"/>
       </om:OMA>
     </om:OMA>
   </om:OMBIND>
 </om:OMOBJ>
</definition>
\end{lstlisting}
\end{erratum}

\noindent The main difference of this definition to the one in the
section above is the {\element[ns-elt=om]{OMOBJ}} element, which now
accompanies the {\element{CMP}} element. It contains a formal
definition of the property of being a law of composition in the form
of a $\lambda$-term $\lambda{E,F}.set(E)\wedge F:E\times E\to
E$\footnote{We actually need to
  import the theories {\snippet{pl1}} for {\twintoo{first-order}{logic}} (it imports
  the theory {\snippet{pl0}}) to legally use the logical symbols here. Since we did not
  show the theory element, we assume it to contain the relevant {\element{imports}}
  elements.}. The value {\attval{simple}{type}{definition}} of the
{\attribute{type}{definition}} attribute in the {\element{definition}} element signifies
that the content of the {\element[ns-elt=om]{OMOBJ}} element can be substituted for the symbol
{\snippet{law\_of\_composition}}, wherever it occurs. So if we have
$\snippet{law\_of\_composition}(A,B)$ somewhere this can be reduced to
$(\lambda{E,F}.set(E)\wedge F:E\times E\to E)(A,B)$ which in turn reduces\footnote{We use
  the $\lambda$-calculus as a formalization framework here: If we
  apply a $\lambda$-term of the form $\lambda{X}.A$ to an argument $B$, then the result is
  obtained by binding all the formal parameters $X$ to the actual parameter $B$, i.e. the
  result is the value of $A$, where all the occurrences of $X$ have been replaced by $B$.
  See~\cite{Barendregt80,Andrews02} for an introduction.} to $set(A)\wedge B:A\times A\to
A$ or in other words $\snippet{law\_of\_composition}(A,B)$ is true, iff $A$ is a set and
$B$ is a function from $A\times A$ to $A$. This definition is directly used in the second
formal definition, which we depict in {\mylstref{magma:formal}}.

\begin{erratum}[reported-by=Michael Kohlhase,date=2009-08-11]{{\texttt{for}} attribute on
    {\texttt{definition}} should be of type {\texttt{NCNames}}}
\begin{erratum}[reported-by=Krzysztof Retel,date=2006-09-14]{should be "definiendum" not "definiens"}
\begin{lstlisting}[label=lst:magma:formal,
    caption={The formal definition of a magma},
    index={definition,CMP,FMP,OMOBJ,OMS,OMA,OMV}]
 <definition xml:id="magma.def" for="magma" type="implicit">
  <CMP> A set with a law of composition is called a 
    <term cd="magmas" name="magma" role="definiendum">magma</term>.
  </CMP>
  <FMP>
    <om:OMOBJ>
      <om:OMBIND><om:OMS cd="pl1" name="forall"/>
        <om:OMBVAR><om:OMV name="M"/></om:OMBVAR>
        <om:OMA><om:OMS cd="pl0" name="iff"/>
          <om:OMA><om:OMS cd="magmas" name="magma"/>
            <om:OMV name="M"/>
          </om:OMA>
          <om:OMBIND>
            <om:OMS cd="pl1" name="exists"/>
            <om:OMBVAR>
              <om:OMV name="E"/><om:OMV name="C"/>
            </om:OMBVAR>
            <om:OMA><om:OMS cd="pl0" name="and"/>
              <om:OMA><om:OMS cd="relation1" name="eq"/>
                <om:OMV name="M"/>
                <om:OMA><om:OMS cd="products" name="Cartesian-product"/>
                  <om:OMV name="E"/>
                  <om:OMV name="C"/>
                </om:OMA>
              </om:OMA>
              <om:OMA><om:OMS cd="magmas" name="law_of_composition"/>
                <om:OMV name="E"/>
                <om:OMV name="F"/>
              </om:OMA>
            </om:OMA>
          </om:OMBIND>
        </om:OMA>
      </om:OMBIND>
    </om:OMOBJ>
  </FMP>
</definition>
\end{lstlisting}
\end{erratum}
\end{erratum}

\noindent Here, the {\attribute{type}{definition}} attribute on the
{\element{definition}} element has the value
{\attval{implicit}{type}{definition}}, which signifies that the
content of the {\element{FMP}} element should be understood as a
logical formula that is made true by exactly one object: the
property of being a {\indextoo{magma}}. This formula can be written
as
\[\allcdot{M}{magma(M)\Leftrightarrow\excdot{E,F}{M=(E,F)\land law\_of\_composition(E,F)}}\]
in other words: $M$ is a magma, iff it is a pair $(E,F)$, where $F$ is a law of
composition over $E$.

Finally, the examples get a formal part as well. This mainly consists of formally
representing the object that serves as the example, and making the way it does
explicit. The first is done simply by adding the object to the example as a
sibling node to the CMP. Note that we are making use of the {\openmath} reference
mechanism here that allows to copy subformulae by linking them with an
{\element[ns-elt=om]{OMR}} element that stands for a copy of the object pointed to by the
{\attribute{href}{OMR}} attribute (see {\mysecref{openmath}}), which makes
this very simple. Also note that we had to split the example into two, since
{\omdoc} only allows one example per {\element{example}} element. However, the
{\element{example}} contains two {\element[ns-elt=om]{OMOBJ}} elements, since the property of
being a law of composition is binary. 

The way this object is an example is made explicit by adding an assertion that makes the
claim of the example formal (in our case that for every set $E$, the function
$(X,Y)\mapsto X\cup Y$ is a law of composition on the set of subsets of $E$). The
assertion is referenced by the {\attribute{assertion}{example}} attribute in the
{\element{example}} element.

\begin{lstlisting}[label=lst:magma-formal-examples,frame=topline,
    caption={A  formalized magma example},
    index={example,assertion,CMP,FMP,OMOBJ,OMS,OMA,OMV}]
<example xml:id="e11.magma" for="law_of_composition" 
         type="for" assertion="e11.magma.ass">
  <CMP> The mapping <om:OMOBJ><om:OMR href="#e11.magma.1"/></om:OMOBJ> is 
    a law of composition on the set of subsets of a set 
    <om:OMOBJ><om:OMS cd="magmas" name="E"/></om:OMOBJ>.
  </CMP>
  <om:OMOBJ>
    <om:OMA id="e11.magma.2"><om:OMS cd="sets" name="subset"/>
      <om:OMV name="E"/>
    </om:OMA>
  </om:OMOBJ>
  <om:OMOBJ>
    <om:OMBIND id="e11.magma.1">
      <om:OMS cd="fns1" name="lambda"/>
      <om:OMBVAR><om:OMV name="X"/><om:OMV name="Y"/></om:OMBVAR>
      <om:OMA>
        <om:OMS cd="functions" name="pattern-defined"/>
        <om:OMA><om:OMS cd="products" name="pair"/>
          <om:OMV name="X"/>
          <om:OMV name="Y"/>
        </om:OMA>
        <om:OMA><om:OMS cd="sets" name="union"/>
          <om:OMV name="X"/>
          <om:OMV name="Y"/>
        </om:OMA>
      </om:OMA>
    </om:OMBIND>
  </om:OMOBJ>
</example>

<assertion xml:id="e11.magma.ass">
  <FMP>
    <om:OMOBJ>
      <om:OMBIND>
        <om:OMS cd="pl1" name="forall"/>
        <om:OMBVAR><om:OMV name="E"/></om:OMBVAR>
        <om:OMA>
          <om:OMS cd="magmas" name="law_of_composition"/>
          <om:OMR href="#e11.magma.2"/>
          <om:OMR href="#e11.magma.1"/>
        </om:OMA>
      </om:OMBIND>
    </om:OMOBJ>
  </FMP>
</assertion>
\end{lstlisting}          
\end{tsection}
\end{tchapter} 
n%%% Local Variables: 
%%% mode: latex
%%% TeX-master: "omdoc" 
%%% End:


% LocalWords:  bourbaki em xsl outerpart Dataset lang  lst utf trl cd dc ref cc
% LocalWords:  def xmlns omstd omcd en nat fns comp pl eq activemath omcds dtd
% LocalWords:  Nicolas setname om dag OMR href openmath mathescape
% LocalWords:  ns attr elt loc nats xy omdoc mbase metadata DOCTYPE omtext CMP
% LocalWords:  Verlag un omgroup OMOBJ OMA OMV mis OMBIND OMBVAR FMP forall
