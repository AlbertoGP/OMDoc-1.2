%%%%%%%%%%%%%%%%%%%%%%%%%%%%%%%%%%%%%%%%%%%%%%%%%%%%%%%%%%%%%%%%%%%%%%%%%
% This file is part of the LaTeX sources of the OMDoc 1.2 specifiation
% Copyright (c) 2006 Michael Kohlhase
% This work is licensed by the Creative Commons Share-Alike license
% see http://creativecommons.org/licenses/by-sa/2.5/ for details
\svnInfo $Id: spec-intro.tex 6154 2006-10-03 11:31:31Z  $
\svnKeyword $HeadURL: https://svn.omdoc.org/repos/omdoc/branches/omdoc-1.2/doc/spec/spec-intro.tex $
%%%%%%%%%%%%%%%%%%%%%%%%%%%%%%%%%%%%%%%%%%%%%%%%%%%%%%%%%%%%%%%%%%%%%%%%%

\begin{tchapter}[id=spec-intro]{OMDoc as a Modular Format}

  A modular approach to design is generally accepted as best practice in the development
  of any type of complex application. It separates the application's functionality into a
  number of "{\indextoo{building blocks}}" or "{\indextoo{module}s}", which are
  subsequently combined according to specific rules to form the entire application. This
  approach offers numerous advantages: The increased {\indextoo{conceptual clarity}}
  allows developers to share ideas and code, and it encourages reuse by creating
  well-defined modules that perform a particular task. Modularization also reduces
  complexity by decomposition of the application's functionality and thus decreases
  debugging time by localizing errors due to design changes. Finally, flexibility and
  maintainability of the application are increased because single modules can be upgraded
  or replaced independently of others.

  The {\omdoc} vocabulary has been split by thematic role, which we will briefly overview
  in {\myfigref{omdoc-modules}} before we go into the specifics of the respective modules
  in {\mychaplref{mobj}{quiz}}. To avoid repetition, we will introduce some attributes
  already in this chapter that are shared by elements from all modules. In
  {\mychapref{document-model}} we will discuss the {\omdoc} document model and possible
  sub-languages of {\omdoc} that only make use of parts of the functionality
  (\mysecref{sub-languages}).

\begin{myfig}{omdoc-modules}{The {\omdoc} Modules}
\begin{small}
\fbox{\begin{tabular}{|l|l|l|l|}\hline
  Module & Title & Required? & Chapter\\\hline\hline
  {\bf\MOBJmodule{spec}} &  Mathematical Objects & yes & {\mychapref{mobj}}\\\hline
    \multicolumn{4}{|p{11cm}|}{\em\footnotesize Formulae are a central part of mathematical
       documents; this module integrates the content-oriented representation
       formats {\openmath} and {\mathml} into {\omdoc}}\\\hline\hline
  {\bf\MTXTmodule{spec}} &  Mathematical Text & yes & {\mychapref{mtxt}}\\\hline
    \multicolumn{4}{|p{11cm}|}{\em\footnotesize Mathematical vernacular,
  i.e. natural language with embedded formulae}\\\hline\hline
  {\bf\DOCmodule{spec}} & Document Infrastructure & yes & {\mychapref{omdoc-infrastructure}}\\\hline
    \multicolumn{4}{|p{11cm}|}{\em\footnotesize  A basic infrastructure for
      assembling pieces of  mathematical knowledge into functional documents and 
      referencing their parts }\\\hline\hline
  {\bf\DCmodule{spec}} & Dublin Core Metadata & yes &   {\mysecsref{dc-elements}{dc-roles}}\\\hline
    \multicolumn{4}{|p{11cm}|}{\em\footnotesize Contains bibliographical ``{\twindef{data}{about data}}'',
      which can be used to annotate many {\omdoc} elements by descriptive and
      administrative information that facilitates navigation and organization}\\\hline\hline 
  {\bf\CCmodule{spec}} & Creative Commons Metadata & yes & {\mysecref{creativecommons}}\\\hline
    \multicolumn{4}{|p{11cm}|}{\em\footnotesize Licenses for text use}\\\hline\hline
  {\bf\RTmodule{spec}} & Rich Text Structure & no & {\mysecref{rt}}\\\hline
    \multicolumn{4}{|p{11cm}|}{\em\footnotesize Rich text structure in
  mathematical vernacular (lists, paragraphs, tables, \ldots)}\\\hline\hline
  {\bf\STmodule{spec}} &  Mathematical Statements & no  & {\mychapref{statements}}\\\hline
    \multicolumn{4}{|p{11cm}|}{\em\footnotesize Markup for mathematical forms like 
      {\indextoo{theorem}s},  {\indextoo{axiom}s}, {\indextoo{definition}s}, 
      and {\indextoo{example}s} that can be used to specify or define properties
      of given mathematical objects and theories to group mathematical
  statements and provide a notion of context.}\\\hline\hline
  {\bf\PFmodule{spec}} &  Proofs and proof objects & no & {\mychapref{proofs}}\\\hline 
    \multicolumn{4}{|p{11cm}|}{\em\footnotesize Structure of proofs
     and argumentations at various levels of details and formality}\\\hline\hline
  {\bf\ADTmodule{spec}} &  Abstract Data Types & no & {\mychapref{adt}}\\\hline 
    \multicolumn{4}{|p{11cm}|}{\em\footnotesize  Definition schemata for
      sets that are built up inductively from constructor symbols}\\\hline\hline 
  {\bf\CTHmodule{spec}} & Complex Theories & no & {\mychapref{complex-theories}}\\\hline
    \multicolumn{4}{|p{11cm}|}{\em\footnotesize Theory morphisms; they can be used
    to structure mathematical theories}\\\hline\hline
  {\bf\DGmodule{spec}} & Development Graphs & no & {\mysecref{development-graphs}}\\\hline
    \multicolumn{4}{|p{11cm}|}{\em\footnotesize Infrastructure for managing theory
  inclusions, change management}\\\hline\hline
  {\bf\EXTmodule{spec}} & Applets, Code, and Data & no & {\mychapref{ext}}\\\hline
    \multicolumn{4}{|p{11cm}|}{\em\footnotesize Markup for applets, program code,
  and data (e.g. images, measurements, \ldots)}\\\hline\hline
  {\bf\PRESmodule{spec}} & Presentation Information & no &  {\mychapref{pres}}\\\hline
    \multicolumn{4}{|p{11cm}|}{\em\footnotesize Limited functionality for
    specifying presentation and notation information for local typographic
      conventions  that cannot be determined by general principles alone}\\\hline\hline
  {\bf\QUIZmodule{spec}} &  Infrastructure for Assessments & no & {\mychapref{quiz}}\\\hline
    \multicolumn{4}{|p{11cm}|}{\em\footnotesize Markup for exercises integrated
    into the {\omdoc} document model}\\\hline 
  \end{tabular}}
\end{small}
\end{myfig}
The first four modules in {\myfigref{omdoc-modules}} are required (mathematical documents
without them do not really make sense), the other ones are optional. The
document-structuring elements in module {\DOCmodule{spec}} have an attribute
{\attributeshort{modules}} that allows to specify which of the modules are used in a
particular document (see {\mychapref{omdoc-infrastructure}} and
{\mysecref{sub-languages}}).

\begin{tsection}[id=omdoc-ns]{The OMDoc Namespaces}
  
  The namespace for the {\omdoc} format is the URI\atwin{OMDoc}{namespace}{URI}
  \url{http://www.mathweb.org/omdoc}. Note that the {\omdoc}
  namespace\twin{OMDoc}{namespace} does not reflect the versions, this is done in the
  {\attributeshort{version}} attribute on the {\twintoo{document}{root}} element
  {\element{omdoc}} (see {\mychapref{omdoc-infrastructure}}).  As a consequence, the
  {\omdoc} vocabulary identified by this namespace is not static, it can change with each
  new {\omdoc} version. However, if it does, the changes will be documented in later
  versions of the specification: the latest released version can be found
  at~\cite{URL:omdocspec}.

  In an {\omdoc} document, the {\omdoc} namespace must be specified either using a
  {\twintoo{namespace}{declaration}} of the form
  {\snippet{xmlns="}}\url{http://www.mathweb.org/omdoc}{\snippet{"}} on the
  {\element{omdoc}} element or by prefixing the {\twintoo{local}{name}s} of the {\omdoc}
  elements by a namespace prefix ({\omdoc} customarily use the prefixes {\snippet{omdoc:}}
  or {\snippet{o:}}) that is declared by a {\atwintoo{namespace}{prefix}{declaration}} of
  the form {\snippet{xmlns:o="}}\url{http://www.mathweb.org/omdoc}{\snippet{"}} on some
  element dominating the {\omdoc} element in question (see {\mysecref{xml}} for an
  introduction). {\omdoc} also uses the following namespaces\footnote{In this
    specification we will use the {\twintoo{namespace}{prefix}es} above on all the
    elements we reference in text unless they are in the {\omdoc} namespace.}:

  \begin{center}\scriptsize
    \begin{tabular}{|l|l|l|}\hline
      Format      & namespace URI & see \\\hline\hline
      Dublin Core & \url{http://purl.org/dc/elements/1.1/} &   {\mysecsref{dc-elements}{dc-roles}}\\\hline
      Creative Commons & \url{http://creativecommons.org/ns} & {\mysecref{creativecommons}}\\\hline
      {\mathml} & \url{http://www.w3.org/1998/Math/MathML} & {\mysecref{cmml}}\\\hline
      {\openmath} & \url{http://www.openmath.org/OpenMath} & {\mysecref{openmath}}\\\hline
      {\xslt} & \url{http://www.w3.org/1999/XSL/Transform} & {\mychapref{pres}}\\\hline
    \end{tabular}
  \end{center}
  Thus a typical document root of an {\omdoc} document looks as follows:
  \begin{lstlisting}[mathescape]
<?xml version="1.0" encoding="utf-8"?>
<omdoc xml:id="test.omdoc" version="1.2"
  xmlns="http://www.mathweb.org/omdoc"
  xmlns:cc="http://creativecommons.org/ns"
  xmlns:dc="http://purl.org/dc/elements/1.1/"
  xmlns:om="http://www.openmath.org/OpenMath"
  xmlns:m="http://www.w3.org/1998/Math/MathML">
$\ldots$
</omdoc>
\end{lstlisting}  
\end{tsection}

\begin{tsection}[id=common-attribs]{Common Attributes in OMDoc}
  Generally, the {\omdoc} format allows any attributes from foreign (i.e. non-{\omdoc})
  namespaces\twin{foreign}{namespace} on the {\omdoc} elements. This is a commonly found
  feature that makes the {\xml} encoding of the {\omdoc} format extensible. Note that the
  attributes defined in this specification are in the default (empty)
  namespace\twin{default}{namespace}\twin{empty}{namespace}: they do not carry a namespace
  prefix. So any attribute of the form {\snippet{na:xxx}} is allowed as long as it is in
  the scope of a suitable {\atwintoo{namespace}{prefix}{declaration}}.
  
  Many {\omdoc} elements have optional {\attributeshort[ns-attr=xml]{id}} attributes that
  can be used as identifiers to reference them. These attributes are of type
  {\snippet{ID}}\twin{type}{ID}, they must be unique in the document which is important,
  since many {\xml} applications\twin{XML}{application} offer functionality for
  referencing and retrieving elements by {\snippet{ID}}-type\twin{type}{ID} attributes.
  Note that unlike other {\snippet{ID}}{\twin{ID}{type}}-attributes, in this special case
  it is the name {\attributeshort[ns-attr=xml]{id}}~\cite{XML:id05} that defines the
  {\indextoo{referencing}} and {\indextoo{uniqueness}} functionality, not the type
  declaration in the {\indextoo{DTD}} or {\twintoo{XML}{schema}} (see
  {\mysubsecref{xml-validation}} for a discussion).

  Note that in the {\omdoc} format proper, all {\twintoo{ID}{type}} attributes are of the
  form {\attributeshort[ns-attr=xml]{id}}. However in the older {\openmath} and {\mathml}
  standards, they still have the form {\attributeshort{id}}. The latter are only
  recognized to be of type {\snippet{ID}}, if a document type or {\xml}schema is
  present. Therefore it depends on the application context, whether a DTD should be
  supplied with the {\omdoc} document.
  
  For many occasions (e.g. for printing {\omdoc} documents), authors want to control a
  wide variety of aspects of the presentation. {\omdoc} is a content-oriented format, and
  as such only supplies an infrastructure to mark up content-relevant information in
  {\omdoc} elements. To address this dilemma {\xml} offers an interface to 
  {\twintoo{Cascading}{Style Sheet}s} ({\css})~\cite{BosHak:css98}, which allow to specify
  presentational traits like {\twintoo{text}{color}}, {\twintoo{font}{variant}},
  {\indextoo{positioning}}, {\indextoo{padding}}, or {\indextoo{frame}s} of
  {\twintoo{layout}{box}es}, and even {\indextoo{aural}} aspects of the text.
  
  To make use of {\css}, most {\omdoc} elements (all that have
  {\attributeshort[ns-attr=xml]{id}} attributes) have {\attributeshort{style}}
  attributes\footnote{The treatment of the {\css} attributes has changed from
    {\omdocv{1.1}}, see the discussion on
    page~\pageref{style/class-comment}.}\twin{CSS}{attribute} that can be used to specify
  {\css} directives\twin{CSS}{directive} for them. In the {\omdoc} fragment in
  {\mylstref{css-basic}} we have used the {\attribute{style}{omtext}} attribute to specify
  that the text content of the {\element{omtext}} element should be formatted in a
  centered box whose width is 80\% of the surrounding box (probably the page box), and
  that has a 2 pixel wide solid frame of the specified RGB color. Generally {\css}
  directives are of the form {\snippet{A:V}}, where {\snippet{A}} is the name of the
  aspect, and {\snippet{V}} is the value, several {\css} directives\twin{CSS}{directive}
  can be combined in one {\attributeshort{style}} attribute as a
  {\twintoo{semicolon-separated}{list}} (see {\cite{BosHak:css98} and the emerging {\css}
    3} standard).

\begin{lstlisting}[label=lst:css-basic,mathescape,
   caption={Basic {\css} Directives in a {\attributeshort{style}} Attribute},
   index={style,class}]
<?xml version="1.0" encoding="utf-8"?>
<?xml-stylesheet type="text/css" href="http://example.org/style.css"?>
<omdoc xml:id="stylish">
  $\ldots$
  <omtext xml:id="t1" style="width:80%;align:center;border:2px #006699 solid">
    <CMP>Here comes something 
      <phrase style="font-weight:bold;color:green" class="emphasize">stylish</phrase>!
    </CMP>
  </omtext>
  $\ldots$
</omdoc>
\end{lstlisting}

Note that many {\css} properties of parent elements are inherited by the children,
if they are not explicitly specified in the child. We could for instance have set
the {\twintoo{font}{family}} of all the children of the {\element{omtext}} element
by adding a directive {\snippet{font-family:sans-serif}} there and then override it by
a directive for the property {\snippet{font-family}} in one of the children.

Frequently recurring groups of {\css} directives can be given symbolic names in
{\css} style sheets\twin{CSS}{style sheet}, which can be referenced by the
{\attributeshort{class}} attribute. In {\mylstref{css-basic}} we have made use of
this with the class {\snippet{emphasize}}, which we assume to be defined in the style
sheet {\snippet{style.css}} associated with the document in the ``{\twintoo{style
    sheet}{processing instruction}}'' in the prolog\footnote{i.e. at the very
  beginning of the {\xml} document before the document type declaration} of the
{\xml} document (see~\cite{Clark:assxd99} for details).  Note that an {\omdoc}
element can have both {\attributeshort{class}} and {\attributeshort{style}}
attributes, in this case, precedence is determined by the rules for {\css}
style sheets as specified in~\cite{BosHak:css98}. In our example in
{\mylstref{css-basic}} the directives in the {\attributeshort{style}} attribute
take precedence over the {\css} directives in the style sheet referenced by the
{\attributeshort{class}} attribute on the {\element{phrase}} element. As a
consequence, the word ``stylish'' would appear in green, bold italics.
\end{tsection}
\end{tchapter}
%%% Local Variables: 
%%% mode: latex
%%% TeX-master: "omdoc"
%%% End: 


% LocalWords:  omdoc mobj attribs DTD XMLSchema dtd css omtext RGB
% LocalWords:  lst emph xml utf stylesheet href DOCTYPE px CMP serif ns dc attr
% LocalWords:  xmlns creativecommons cmml openmath cc om mtxt rt adt ext pres
% LocalWords:  na xxx es sans mathescape prolog
